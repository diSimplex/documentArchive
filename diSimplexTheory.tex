% LaTeX source for the diSimplexTheory document
%

\documentclass[a4paper,openany]{amsbook}
\usepackage{disitt}
\usepackage{disitt-symbols}
\usepackage{mdframed}
\newmdenv[linecolor=white,backgroundcolor=gray!10]{infobox}
\newenvironment{myQuote}{\begin{quotation}}{\end{quotation}}
\surroundwithmdframed[linecolor=white,backgroundcolor=gray!10]{myQuote}

\begin{document}
\frontmatter
\sloppy

\title[DiSimplicial Theory]{Reality of Mathematics: Directed Simplicial Theory, A 
categorical trans-finite computational foundation for Mathematics}
%% author: stg
\author{Stephen Gaito}
\address{PerceptiSys Ltd, 21 Gregory Ave, Coventry, CV3 6DJ, United Kingdom}%
\email{stephen@perceptisys.co.uk}%
\urladdr{http://www.perceptisys.co.uk}


%% version summary
\thanks{Created: 2014-05-13}
\thanks{Git commit \gitReferences{} (\gitAbbrevHash{}) commited on \gitAuthorDate{} by \gitAuthorName{}}
\thanks{AMS-\LaTeX{}'ed on \today{}.}

%% Copyrights
\thanks{\textbf{Copyright: \copyright{} Stephen Gaito, PerceptiSys Ltd \the\year{}; Some rights reserved}}
\thanks{\textbf{This work is licensed under a Creative Commons Attribution-ShareAlike 4.0 International License.}}

\subjclass[2010]{Primary unknown; Secondary unknown} %
\keywords{Keyword one, keyword two etc.}%

\begin{abstract}
To be able to provide a mathematical theory of Reality, one must first 
address the question: \textit{How ``Real'' is Mathematics?}
\end{abstract} 
\maketitle 
\tableofcontents 
\mainmatter

\begin{quotation}
In the places I go there are things that I see\\
That I \emph{never} could spell if I stopped with the Z.\\
\begin{quote}
\textit{On Beyond Zebra} by Dr Seuss.
\end{quote}
\end{quotation}

\section{Introduction}

The aim of this cycle of papers is to provide a rigorous mathematical theory of
Reality. This first paper provides \emph{an} answer to the question: \emph{How
`Real' is Mathematics?}, that is, \emph{what is the validity of the
mathematics used to provide a mathematical theory of Reality?}

Any answer to this question, which forms part of this cycle of papers, must of
necessity be circular: \emph{what is the Reality of the beings who consider the
validity of the mathematics which provides a mathematical description of
Reality?} To break this circularity, we take the pragmatic approach,
bastardising Descartes, and state: \emph{I make marks in the sand, so I exist}.

Any discussion of the `Reality' or `Validity' of Mathematics is essentially a
discussion of the \textit{Foundations of Mathematics}. Classically Mathematics
is founded upon some form of Set Theory, such as Zermelo-–Fraenkel set theory
with the axiom of choice (ZFC), expressed in the language of First Order
Predicate Logic. To avoid Russel's paradox, one form of the Liar Paradox, ZFC
explicitly includes, amongst others, an axiom forcing all sets to be
well--founded. From a Categorical point of view, this means that classical Set
Theory is an \emph{Algebraic} theory. From a computational point of view, this
means that classical Set Theory is `data'.

One of the primary aims of this cycle of papers, is to show that the Sciences
and Engineering, are better served by a Mathematics which focuses on spatially
distributed interacting \emph{processes}. This means that, from a Categorical
point of view, we are interested in \emph{Co-Algebraic} structures as the
Foundations of Mathematics.

Equally importantly, `Truth', as expressed by Aristotelian Logic, in the form of
First Order Predicate Logic, is less important then the ability to compute. We
will show that First Order Predicate Logic, as traditionally used in
Mathematics, is the \emph{internal} logic of \emph{the} Co-Algebraic Universe of
Mathematics. We will refer to this Co-Algebraic Universe as \emph{`Plato's
Universe'}, \Universe{}{}.

\begin{myQuote}
Put simply, we base our foundations of mathematics on generalised
computation \emph{rather than} Logic.
\end{myQuote}

Clearly, not all of classical Mathematics is (classically) `computable'. For
example, the use of the (the unrestricted) Axiom of Choice, is un-computable by
a classical Turing Machine. This suggests that, to capture the whole of
classical Mathematics, we will need a more nuanced model of computation. The
critical point here is that classical mathematics is developed from an
\emph{omnipotent} point of view, yet as finite beings, we only have far more
limited computational abilities. In our best 'school Latin`, we christen beings
with these more limited abilities, as \emph{modipotent}.

The classical theory of computation typically investigates the relative power of
computation using turing machines \emph{extended} with oracles. We will instead introduce
computation whose power is controlled by the algebraic `size' of its available data
structures and/or the amount of `time' available for the computation. For each
\emph{algebraic} cardinal, $\gamma$, there is an associated computational power of a
$\gamma$-modipotent being, or $\gamma$-being. The unlimited computational power of an
omnipotent being (a `god') corresponds to the \emph{co-algebraic} cardinal, \Cardinal.
Hence any omnipotent being will be referred to as a \Cardinal-being.

The co-algebraic \emph{sub}universe of mathematics computable by an
$\omega$-modipotent being, such as ourselves, will be referred to as
\emph{`Plato's Playground'}.

One of our primary interests is in how \emph{`Plato's Playground'} fits `inside'
\emph{`Plato's universe'}. Clearly, $\omega$-modipotent beings can `dream' of,
and hence specify, structures which they can not construct using
$\omega$-computational tools. That is there are structures whose specifications
`exist' in Plato's playground but which can only be realised in Plato's
universe.

Often to really understand a topic, it is useful to develop the same ideas using very
different tools. After just over 100 years, it is past time to develop the foundations of
mathematics using different tools.  In our case using computation rather than logic.
However, not surprisingly, there will be rough correspondences between the different sets
of tools. Again, in our case, each basic assumption in our computational theory of
mathematics, corresponds to one or more of the axioms of Set Theory. The essentiality of
these correspondences is what provides us with a \emph{deep} understanding of what it
\emph{means} to \emph{found} mathematics.

How can we understand the shift in focus from a Logical foundation for
mathematics to a Computational foundation for mathematics? In the classical
Logical foundation of mathematics it is the study of the (algebraic) structure
of the logical implications of a set of axioms which is primary. Given these
sets of axioms we can then investigate the collection of \emph{models} which
satisfy the axioms. In the Computational foundations of mathematics, it is the
co-algebraic collections of \emph{constructed} models which is primary.  Given
these co-algebraic models there are then logical statements which distinguish,
test or verify the models.

\TODO{rework the following}

As will become obvious later in this cycle of papers, ``Reality'' is \emph{a}
model of \emph{a} higher topos.  Unfortunately, for finite beings \emph{inside}
Reality, we can only ever specify a possible collection of possible Realities. A
collection of Realities which are consistent with our current, finitely limited,
understanding, or measurement of ``Reality''.

Far more fundamental and far more important than propositions is the concept of
Causality.  I will argue, in later work, that the human brain, is a causality
inference engine. I will equally argue that the unification of Relativity and
Quantum Mechanics requires a careful understanding of the \emph{finite}
structure of causality.

As should become evident, developing the mathematics of the \emph{finite}
structure of causality, is most easily done when based upon Directed Simplicial
Theory rather than classical propositional/predicate calculus. Following the
lead of HoTT, propositional/predicate calculus, as well as set theory will be
specific ``universes'' of Directed Simplicial Type Theory.

The important point here, is that, while the mathematics of causality is
\emph{implicitly} ``part'' of classical logic, it is conceptually useful to make
the structure of causality \emph{explicit} in our foundations of mathematics. 
Inferring and manipulating causality is after all the main aim of all of the
sciences and engineering.

This book is devoted to giving the reader a more ``mathematical'' overview of
Directed Simplicial Theory.  It is meant to be read in companion with the very
much more detailed \textit{Directed Simplicial Theory Implemented in Haskell}.

\section{Building Plato's Wilderness}

\begin{myQuote}
Our explicit thesis is that Mathematics \emph{is} computation.
\end{myQuote}

To begin, we must provide a model of computation. The classical models of
computability, such as Turing Machines, the Lambda Calculus and General
Recursive Functions, are all, more or less, about computational sequentiality
and `time'. While the model we are about to propose will be equivalent to any of
the classical models, it is designed to be \emph{explicitly} about parallel
computation and `space--time'. Moreover, since classical mathematics is
mathematics from an omnipotent point of view, the computational model required
to capture classical mathematics, must \emph{explicitly} have unbounded
computational ability.

Since Mathematics \emph{is} computation, we, as mathematicians, write \emph{in}
a programming language. As in Computer Science, there are many programming
languages, each suited to a different paradigm, there will be numerous
mathematical languages.

Again, in Theoretical Computer Science, given a programming language, the first question
is: \emph{How do we know what this language computes?} This question is captured by the
Axiomatic, Denotational and Operational Semantics associated with a given language.
Plato's universe, \emph{the} Co-Algebraic universe of Mathematics, is essentially, the
fixed point of these Semantic descriptions.

Equally importantly, since `proof' is computation, how do we know that a `theorem' is
`correct'. In Theoretical Computer Science, proofs that a given programaic artefact
computes what it is meant to compute, are done using Axiomatic Semantics. Axiomatic
Semantical proofs that a Theorem is correct are what we finite beings can use to show that
Theorems which require trans-finite resources correctly compute the desired result. Given
that we do not have classical `logic' at our disposal, we will use proof theoretic rules
engine to build and verify these Axiomatic Semantical proofs.

When the dust settles, to define Plato's Universe, \Universe{}{}, we will need to
simultaneously define, via co--induction, the four structures of \Universe{}{},
\Lists, \ListAutomorphisms, and the proof theoretical rule engine:
%
\begin{align}
   \Universe{}{}          & \isomorphic \arrow{\ListAutomorphisms}{\Universe{}{}} & 
   \text{Plato's Universe} \\
   \Lists             & \isomorphic \one + \Universe{}{} \times \Lists        & 
   \text{(Gernalized) Lists} \\
   \ListAutomorphisms & \isomorphic \arrow{\Lists}{\Lists}                & 
   \text{(Gernalized) List Automorphisms} \\
    &  & \text{Rules Engine}
\end{align}
%
Unfortunately, since we \emph{are} building the foundations of Mathematics, we,
as yet, have no definitions of co--induction, $\isomorphic$, \one, $+$, $\times$,
$\arrow{\cdot}{\cdot}$, let alone, \Universe{}{}, \Lists\ or even \ListAutomorphisms.

Again, when the dust settles, Lists, \Lists, is \emph{an} \emph{implementation}
of the classical concept of the Ordinals, \Ordinal, and the isomorphism objects
of the Lists modulo List Automorphisms, \ListAutomorphisms, is \emph{an}
\emph{implementation} of the classical concepts of the Cardinals, \Cardinal. In
deed we will, eventually, make these explicit definitions, however, at the
moment, we are getting ahead of our current (minimal) abilities.

\subsection{Syntax: Marks in the sand}

To be able to discuss the mathematics of an omnipotent \Cardinal-being, we
\emph{explicitly} place \emph{no limits} on the size of the syntax. This means we
make no assumption about the numbers of symbols which can be used, the number of
symbols which can be `written', nor the length of time which might be used to
`write these symbols down', since, of course, an omnipotent \Cardinal-being has
no such limits. In each case we explicitly allow for a `transfinite number of
symbols', a `transfinite sized piece of paper', and a `transfinite amount of
time' (when of course we have suitably defined the `transfinite ordinals'). Note
that, in contrast to classical set theory, to `compute' any large ordinal,
instead of merely assuming it exists, an omnipotent \Cardinal-being does indeed
require large ordinal lengths of time to write down large ordinal numbers of
symbols.

DiSimplicial structures, which are instances of Plato's universe, are highly
structured objects with complex referential structure. There are two
distinct types of references that we will use: 
%
\begin{itemize}
%
\item Specific known references, \define{names}{}, used to cross-link known
(sub)structures in a given diSimplicial structure, and 
%
\item Specific unknown references, \define{variables}{}, used to denote
(sub)structures we might be in the process of building and/or which we assume
might exist. 
%
\end{itemize}
%

Since we are, among other things, developing a theory of computation, we follow
Computer Scientists by using capital ASCII letters to denote \emph{names} and
lower case ASCII letters to denote \emph{variables}. However, since we need a
way of transcribing a diSimplicial structure as used by a possibly omnipotent
\Cardinal-being, and since such a being might use more than the finite number of
ASCII letters, we need a way of producing a transfinite number of symbols. To do
this we assume that there is a well know and fixed numeration associated with
any (co-)algebraic ordinal, then we can use any such numeral as a subscript of a
given denotation of a name or variable.

We proceed by providing a Backus-Naur Form (BNF) description of the allowed
denotations of names and variables\footnote{We use an extended Backus-Naur form
by allowing the use of standard POSIX regular expressions.}. Each $\gamma$-being
will potentially have a different numeration. Our BNF for the syntax of a
diSimplicial structure allows for alternate numerations.

\newcommand{\bnf}[2]{\ensuremath{<\textbf{#1}_{#2}>}}
\newcommand{\bnfAssign}{\text{::=}}
\newcommand{\bnfT}[1]{\text{`}#1\text{'}}

For an $\omega$-being a potential numeration is:
%
\begin{align*}
%
\bnf{numeral}{\omega} \quad & \bnfAssign & \text{[1-9]}\text{[0-9]*}
%
\end{align*}
%

For an $\epsilon_0$-being we can use the numeration based upon Cantor's normal
form:
%
\begin{align*}
%
\bnf{numeral-part}{\epsilon_0} \quad & \bnfAssign & 
  \bnfT{\omega} \bnf{numeral}{\epsilon_0} \bnfT{\cdot} \bnf{numeral}{\omega} \\
%
\bnf{numeral}{\epsilon_0} \quad & \bnfAssign & 
  \bnf{numeral-part}{\epsilon_0} ( \; \bnfT{+} \bnf{numeral-part}{\epsilon_0} ) \text{*}
%
\end{align*}
%
where we have redundant representations unless the exponents of $\omega$ are
strictly ordered $\beta_1, > \ldots > \beta_k \ge 0$.

We postpone a potential numeration for a \Cardinal-being until after we have
defined the full syntax for diSimplicial structures.

In the following BNF, we use \bnf{numeral}{} to denote any given fixed
numeration.

For any being, names and variables are defined by:
%
\begin{align*}
%
\bnf{nv-part}{} \quad & \bnfAssign & 
  \textbf{[A-Za-z0-9]+} \;\; ( \bnfT{\underline{\hspace{1ex} }} \bnf{numeral}{} )\text{?} \\
%
\bnf{name}{} \quad & \bnfAssign & \textbf{[A-Z]} \bnf{nv-part}{} \\
%
\bnf{variable}{} \quad & \bnfAssign & \textbf{[a-z]} \bnf{nv-part}{}
%
\end{align*}

We begin by developing the syntax for \Universe{}{1}.
%
\begin{align*}
%
\bnf{diSimplex}{-1} \quad & \bnfAssign & \bnfT{\Delta_{-1}} \;\; \bnfT{\langle} \;\; \bnfT{\rangle} \\
%
\bnf{diSimplex}{0} \quad & \bnfAssign & \bnfT{\Delta_0} \;\; \bnfT{\langle} \bnf{name}{} \bnfT{\rangle} \\
%
\bnf{diSimplex}{1} \quad & \bnfAssign & 
  \bnfT{\Delta_1} \;\; \bnfT{\langle} \bnf{name}{}, \bnf{name}{} \bnfT{\rangle}
%
\end{align*}
%
where the structures referred to by a name in each of $\Delta_0$ and $\Delta_1$
constructs must be $\Delta_{-1}$ and $\Delta_0$ structures respectively.

\subsection{new work}

We ``build'' the rules for Plato's universe, \Universe{}{}, by simultaneously providing rules
for construction, bi-simulation, and modal logic.

When considering both the rules for bi-simulation and modal logic, it is important to
realise that these rules apply to diSimplicial \emph{structures} and not to ``individual''
simplicies in a given structure. Effectively a fragment of modal logic provides a
specification of a ``subset'' of all diSimplicial structures which conform to that
fragment of modal logic.

A diSimplicial structure is a ``global'' ``object''.  It is made up of ``local'' simplicies. 
Computation is ``local''. This means it proceeds by ``walking'' over the simplicies 
``contained'' ``in'' a given diSimplicial structure. However, ``locality'' depends upon the 
relative power of a given being. ``Locality'' for an $\omega$-being (or $\aleph_0$-being) 
means finite collections of simplicies. ``Locality'' for an $\aleph_1$-being means countable 
collections of simplicies. For an omnipotent \Cardinal-being, ``locality'' and ``globality'' 
coincide.

It is equally important to realise that we take an explicitly \emph{structuralist}
approach. An ``object'' which has \emph{no} relationships with anything else, does not
exist (as far as ``everything else'' is concerned). This is essentially a mathematical 
version of Mach's principle. Local structure is determined by global relationships.  This 
approach \emph{determines} our rules for bi-similarity.

We begin at the beginning\footnote{We loosely follow the presentations found in the HoTT
book, \cite[Appendix A2]{ufp2013hott} and Bart Jacobs' \cite{jacobs1999catLogicTypeTh},
but of course with our own interpretations and deviations.}:

A \define{judgement}{} is a reference to one or more computational structures in Plato's
universe. A judgement might either represent an assumption that such a structure exists,
or be the result of a computation.

A \define{context}{}, \cJudgement{\Gamma}, is a collection of the currently accepted
judgements. As such a context encodes the explicitly known relationships between the
relevant parts of a collection of diSimplicial structures. A context is itself a
judgement.

A \define{part of judgement}{}, \judgement{b}{B}, is the statement that one diSimplicial 
structure, $b$, is a (sub)part of another one, $B$.

At the foundational level, these are the \emph{only} judgements. All other judgements are
derived from these two.

A \define{contextual judgement}{}, $\context{\Gamma} \vdash \mathcal{J}$, is the assertion 
that, in the context, \context{\Gamma}, we can compute the judgement, $\mathcal{J}$. Note 
that $\mathcal{J}$ can be either a context judgement, \cJudgement{\Gamma}, or a part of 
judgement, \judgement{a}{A}, or any derived judgement.

These and all other judgements are essentially defined by the collection of
\define{inference rules}{} in which they participate. A typical inference rule, named,
Name, has the form:
%
\begin{prooftree}
\AxiomC{$c\mathcal{J}_0$}
\AxiomC{$\cdots$}
\AxiomC{$c\mathcal{J}_{\gamma}$}
\RightLabel{Name}
\TrinaryInfC{$c\mathcal{J}$}
\end{prooftree}
%
For some ordinal, $\gamma$, and contextual judgements, $c\mathcal{J}_0, \cdots, 
c\mathcal{J}_{\gamma}$ and $c\mathcal{J}$. 

Following, \cite[section 2.5]{sangiorgi2012introBisimulationCoinduction}, we can read any
rule in either the forward or backward direction.  In the forward direction a rule helps
to define an algebra by showing one way in which an ``object'' is built up out of its
parts. In the backward direction a rule helps to define a coalgebra by showing one way in
which to observe the parts of an ``object''.  We will make explicit use of \emph{both} the
forward and backward readings of any and all rules.

\subsection{Context rules}

\TODO{Translate these rules into disitt-engine format; redefine them for transfinite use.}

\begin{prooftree}
\AxiomC{}
\RightLabel{empty-ctx}
\UnaryInfC{\cJudgement{\cdot}}
\end{prooftree}

\begin{prooftree}
\AxiomC{$\context{\mathcal{\hat{J}}} \vdash \mathcal{J}$}
\RightLabel{next-ctx}
\UnaryInfC{\cJudgement{\mathcal{\hat{J}}, \mathcal{J}}}
\end{prooftree}

As any self respecting functional programmer will recognise, this is a twisted,
\emph{co-algebraic}, definition of a list. Instead of growing to the right, this list
structure grows to the left. As we will see below, the pair of inference rules, empty-ctx
and next-ctx together imply that a context is a mapping from the \emph{computable}
ordinals into the collection of judgements. For omnipotent beings, the computable ordinals
\emph{are} the ordinals.

\begin{prooftree}
\AxiomC{$\context{\mathcal{J}_0, \ldots, \mathcal{J}_n} \vdash \mathcal{J}_{n+1}$}
\RightLabel{successor-ctx}
\UnaryInfC{\cJudgement{\mathcal{J}_0, \ldots, \mathcal{J}_n, \mathcal{J}_{n+1}}}
\end{prooftree}

\begin{prooftree}
\AxiomC{$\context{\mathcal{J}_0, \ldots, \mathcal{J}_n} \vdash 
\cJudgement{\mathcal{J}_{n+1}, \ldots, \mathcal{J}_{n+m}}$}
\RightLabel{flatten-1-ctx}
\UnaryInfC{$\context{\mathcal{J}_0, \ldots, \mathcal{J}_n, \mathcal{J}_{n+1}} \vdash 
\cJudgement{\mathcal{J}_{n+2}, \ldots, \mathcal{J}_{n+m}}$}
\end{prooftree}

\begin{prooftree}
\AxiomC{$\context{\mathcal{J}_0, \ldots, \mathcal{J}_n} \vdash 
\cJudgement{\cJudgement{\mathcal{J}_{n+1,0},\ldots,\mathcal{J}_{n+1,k}}, \ldots, 
\mathcal{J}_{n+m}}$}
\RightLabel{flatten-2-ctx}
\UnaryInfC{$\context{\mathcal{J}_0, \ldots, \mathcal{J}_n, \mathcal{J}_{n+1,0}} \vdash 
\cJudgement{\cJudgement{\mathcal{J}_{n+1,1},\ldots,\mathcal{J}_{n+1,k}}, \ldots, 
\mathcal{J}_{n+m}}$}
\end{prooftree}

\TODO{I now assert that the two flattens above provide sufficient computational power to
compute the total flattening of any countably structured context.... PROVE THIS.
Essentially given these rules, a context is a countably infinitely branching countably
infinitely deep tree. To flatten it we need to follow a weaving breadth first pattern
continuously revisiting nodes until in the ``countably inifinite time'' it is flattened. 
This \emph{will} be a typcial problem/solution for diSimplicial structures. Essentially this 
is nothing more nor less than a \emph{fair} parallel scheduling algorithm for the parallel 
application of, in this case, the flattening rules on all branches simultaneously. }

\subsection{\Universe{}{-1}, \Universe{}{0} and \Universe{}{1}}

See \cite{bell1988topoiSetThLogic, joyalMoerdijk1995algSetTh, aczel1988nonWellFoundedSets}

\begin{prooftree}
\AxiomC{\cJudgement{\Gamma}}
\RightLabel{empty-\Universe{}{}}
\UnaryInfC{$\context{\Gamma} \vdash \judgement{\cdot}{\Universe{}{}}$}
\end{prooftree}

\begin{prooftree}
\AxiomC{$\context{\Gamma} \vdash \judgement{a}{\Universe{}{}}$}
\RightLabel{reflexive-partOf}
\UnaryInfC{$\context{\Gamma} \vdash \judgement{a}{a}$}
\end{prooftree}

\begin{prooftree}
\AxiomC{$\context{\Gamma} \vdash \judgement{a}{b}$}
\AxiomC{$\context{\Gamma} \vdash \judgement{b}{a}$}
\RightLabel{antisymmetry-partOf}
\BinaryInfC{$\context{\Gamma} \vdash a =_{\Universe{}{}} b$}
\end{prooftree}

\begin{prooftree}
\AxiomC{$\context{\Gamma, \judgement{b}{a}} \vdash \judgement{c}{b}$}
\RightLabel{tansitive-partOf}
\UnaryInfC{$\context{\Gamma} \vdash \judgement{c}{a}$}
\end{prooftree}

\begin{prooftree}
\AxiomC{$\context{\Gamma} \vdash \judgement{n}{\Universe{}{}}$}
\RightLabel{empty-\Universe{}{0}}
\UnaryInfC{$\context{\Gamma} \vdash \judgement{\Delta_{-1}(n)}{\Universe{}{-1}}$}
\end{prooftree}

\begin{prooftree}
\AxiomC{$\context{\Gamma} \vdash \judgement{a}{\Universe{}{-1}}$}
\RightLabel{empty-\Universe{}{0}}
\UnaryInfC{$\context{\Gamma} \vdash \judgement{a}{\Universe{}{}}$}
\end{prooftree}

\begin{prooftree}
\AxiomC{$\context{\Gamma} \vdash \judgement{n}{\Universe{}{}}$}
\AxiomC{$\context{\Gamma} \vdash \judgement{a}{\Universe{}{-1}}$}
\RightLabel{empty-\Universe{}{0}}
\BinaryInfC{$\context{\Gamma} \vdash \judgement{\Delta_0(n;a)}{\Universe{}{0}}$}
\end{prooftree}

\begin{prooftree}
\AxiomC{$\context{\Gamma} \vdash \judgement{a}{\Universe{}{0}}$}
\RightLabel{empty-\Universe{}{0}}
\UnaryInfC{$\context{\Gamma} \vdash \judgement{a}{\Universe{}{}}$}
\end{prooftree}

\begin{prooftree}
\AxiomC{$\context{\Gamma} \vdash \judgement{n}{\Universe{}{}}$}
\AxiomC{$\context{\Gamma} \vdash \judgement{a}{\Universe{}{0}}$}
\AxiomC{$\context{\Gamma} \vdash \judgement{b}{\Universe{}{0}}$}
\RightLabel{empty-\Universe{}{0}}
\TrinaryInfC{$\context{\Gamma} \vdash \judgement{\Delta_1(n;a,b)}{\Universe{}{1}}$}
\end{prooftree}

\begin{prooftree}
\AxiomC{$\context{\Gamma} \vdash \judgement{a}{\Universe{}{1}}$}
\RightLabel{empty-\Universe{}{0}}
\UnaryInfC{$\context{\Gamma} \vdash \judgement{a}{\Universe{}{}}$}
\end{prooftree}

\subsection{Definition of equality}

The first and most important thing we must do is to, given the above definitions of
Plato's universe, \Universe{}{}, define \define{equality}{}.  That is we need to show that
we can compute the following:

\begin{prooftree}
\AxiomC{\hypothesis{\Gamma}{\judgement{a}{\Universe{}{0}}}}
\AxiomC{\hypothesis{\Gamma}{\judgement{b}{\Universe{}{0}}}}
\AxiomC{\hypothesis{\Gamma}{\judgement{a}{b}}}
\AxiomC{\hypothesis{\Gamma}{\judgement{b}{a}}}
\QuaternaryInfC{\conclusion{\Gamma}{\judgement{a = b}{\Universe{}{0}}}}
\end{prooftree}

\begin{prooftree}
\AxiomC{\hypothesis{\Gamma}{\judgement{a}{\Universe{}{1}}}}
\AxiomC{\hypothesis{\Gamma}{\judgement{b}{\Universe{}{1}}}}
\AxiomC{\hypothesis{\Gamma}{\judgement{a}{b}}}
\AxiomC{\hypothesis{\Gamma}{\judgement{b}{a}}}
\QuaternaryInfC{\conclusion{\Gamma}{\judgement{a = b}{\Universe{}{1}}}}
\end{prooftree}

\subsection{\Universe{}{} is a Category}

Given the above (partial) definition of Plato's universe, \Universe{}{}, our first
intention is to show that it is a Category in the classical definition.  It is critical to
note that we are explicitly assuming omnipotent, unlimited, computing power.

To show that \Universe{}{} is a Category we need to show that we can compute the following 
rules:

\begin{prooftree}
\AxiomC{\hypothesis{\Gamma}{\judgement{f}{\Universe{}{1}}}}
\AxiomC{\hypothesis{\Gamma}{\judgement{g}{\Universe{}{1}}}}
\AxiomC{\hypothesis{\Gamma}{\judgement{\pi_0(f) = \pi_1(g)}{\Universe{}{0}}}}
\TrinaryInfC{\conclusion{\Gamma}{\judgement{f \circ g}{\Universe{}{1}}}}
\end{prooftree}

\begin{prooftree}
\AxiomC{\hypothesis{\Gamma}{\judgement{f}{\Universe{}{1}}}}
\AxiomC{\hypothesis{\Gamma}{\judgement{g}{\Universe{}{1}}}}
\AxiomC{\hypothesis{\Gamma}{\judgement{h}{\Universe{}{1}}}}
\AxiomC{\hypothesis{\Gamma}{\judgement{\pi_0(f) = \pi_1(g)}{\Universe{}{0}}}}
\AxiomC{\hypothesis{\Gamma}{\judgement{\pi_0(g) = \pi_1(h)}{\Universe{}{0}}}}
\QuinaryInfC{\conclusion{\Gamma}{\judgement{((f \circ g) \circ h) = (f \circ (g \circ h)) 
}{\Universe{}{1}}}}
\end{prooftree}

\begin{prooftree}
\AxiomC{\hypothesis{\Gamma}{\judgement{a}{\Universe{}{0}}}}
\UnaryInfC{\conclusion{\Gamma}{\judgement{\one_a}{\Universe{}{1}}}}
\end{prooftree}

\begin{prooftree}
\AxiomC{\hypothesis{\Gamma}{\judgement{\one_a}{\Universe{}{1}}}}
\AxiomC{\hypothesis{\Gamma}{\judgement{f}{\Universe{}{1}}}}
\AxiomC{\hypothesis{\Gamma}{\judgement{\pi_0(\one_a) = \pi_1(f)}{\Universe{}{0}}}}
\TrinaryInfC{\conclusion{\Gamma}{\judgement{(f \circ \one_a) = f}{\Universe{}{1}}}}
\end{prooftree}

\begin{prooftree}
\AxiomC{\hypothesis{\Gamma}{\judgement{\one_a}{\Universe{}{1}}}}
\AxiomC{\hypothesis{\Gamma}{\judgement{f}{\Universe{}{1}}}}
\AxiomC{\hypothesis{\Gamma}{\judgement{\pi_0(f) = \pi_1(\one_a)}{\Universe{}{0}}}}
\TrinaryInfC{\conclusion{\Gamma}{\judgement{(\one_a \circ f)= f}{\Universe{}{1}}}}
\end{prooftree}

\subsection{(Von Neumann) Ordinals}

See \cite{joyalMoerdijk1995algSetTh, aczel1988nonWellFoundedSets}

\begin{prooftree}
\AxiomC{$\context{\Gamma} \vdash \judgement{\cdot}{\Universe{}{}}$}
\RightLabel{empty-\Universe{}{0}}
\UnaryInfC{$\context{\Gamma} \vdash \judgement{\cdot}{\Ordinal}$}
\end{prooftree}

\begin{prooftree}
\AxiomC{$\context{\Gamma} \vdash \judgement{n}{\Ordinal}$}
\RightLabel{empty-\Universe{}{0}}
\UnaryInfC{$\context{\Gamma} \vdash \judgement{\Delta_0(n; 
\Delta_{-1}(n))}{\Ordinal}$}
\end{prooftree}

The following are both computable:

\begin{prooftree}
\AxiomC{$\context{\Gamma} \vdash \judgement{n}{\Ordinal}$}
\RightLabel{empty-\Universe{}{0}}
\UnaryInfC{$\context{\Gamma} \vdash \judgement{n}{\Universe{}{}}$}
\end{prooftree}


\begin{prooftree}
\AxiomC{$\context{\Gamma} \vdash \judgement{n}{\Ordinal}$}
\RightLabel{empty-\Universe{}{0}}
\UnaryInfC{$\context{\Gamma} \vdash \judgement{\Delta_1(\cdot; n,\Delta_0(n; 
\Delta_{-1}(n)))}{\Universe{}{1}}$}
\end{prooftree}

\subsection{Denotational Semantics}



\subsection{Operational Semantics}



\begin{definition}
content...
\end{definition}

\subsection{motivation}

A sketch of nearly random ideas relating to the base of the foundations of
mathematics via computational complexity.

start by looking at finite computation as above BUT we need to generalize lists
to encompass the supremum operator. See chapter 2 and 3 of Algebraic Set Theory.
compare the trees used to prove the existence of ZF-Algebras (chapter 3) with
those used to help motivate Aczel's Non-Well-Founded sets. compare these with
Vopenka's (rigid graph/tree) principle in Adamek and Rosicky's book Locally
Presentable and Accessible Categories.

Vopenka's principle is described in Mathoverflow articles:
\verb|http://mathoverflow.net/questions/29302/reasons-to-believe-vopenkas-principle-huge-cardinals-are-consistent/29473#29473|
and
\verb|http://mathoverflow.net/questions/45602/can-vopenkas-principle-be-violated-definably/46538#46538|
and or Adamek and Rosicky's book as really delineating the ``correct'' boundary
between small and large sets. As such our theory \emph{should} ensure the
Vopenka's principle is honoured.

basically using AST's existence work, we can build the ordinals just in time (?)
to extend finite computers into transfinite computers. (See Moschovakis's paper
on definition of algorithms -- he does not go far enough and use co-algebraic
methods... but his computer tools do assume ordinal computation as we need to
do).

need to understand the relationship between implicit and explicit computation
--- see the similar dichotome in the lambda-calculus.

look for p =!= np in the categorical theorems which force the use of the
small/large boundary.

look at: lucas2001transfiniteRewrite,
dershowitzKaplanPlaisted1989transfiinteRewrite,
jacobs2011coalgebraicComputation

\bibliographystyle{amsalpha}
\bibliography{diSimplexTheory}

\end{document}

