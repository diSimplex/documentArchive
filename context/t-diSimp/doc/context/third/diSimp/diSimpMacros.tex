% A ConTeXt document [master document: diSimp.tex]

\chapter[title=diSimplex macros]

\startMkIVCode

% see: https://www.contextgarden.net/Rsfs
%
% unfortunately this is mkii style ;-(
%  the \definefamilysynonym has changed but I can not find out the
%  mkiv equivalent
%
%\starttypescript [math] [modern,computer-modern,latin-modern,ams] [size]
%     \definebodyfont [17.3pt,14.4pt,12pt,11pt,10pt,9pt][mm][mc=rsfs10 sa 1]
%     \definebodyfont [8pt,7pt] [mm] [mc=rsfs7 sa 1]
%     \definebodyfont [6pt,5pt,4pt] [mm] [mc=rsfs5 sa 1]
%\stoptypescript
%
%\definefamilysynonym [default] [scriptfamily] [mc]
%
%\def\scr{\fam\purefamily{scriptfamily}} % To use ${\scr ABC}$
%\def\script#1{{\scr#1}}                 % To use $\script{ABC}$

% see:  https://wiki.contextgarden.net/math_calligraphic
% NOW included in context-experimental..
%
%\definefallbackfamily
%  [mainface]
%  [math]
%  [Asana Math]
%  [range={uppercasescript,lowercasescript}]
%
%\definefontfamily [mainface] [rm] [Latin Modern]
%\definefontfamily [mainface] [math] [Latin Modern Math]
%
%\setupbodyfont[mainface]
%
%\starttext
%${\cal ABCDEFGHIJKLMNOPQRSTUVWXYZ}$
%
%${\cal abcdefghijklmnopqrstuvwxyz}$
%
%The space of distributions on $\Omega$ is denoted by ${\cal D}'(\Omega)$, while ${\cal S}({\Bbb R}^n)$ is the space of Schwartz functions, and ${\cal S}'({\Bbb R}^n)$ is its dual.
%
%\startformula
%\langle T,\phi \rangle_{{\cal D}'(\Omega),{\cal D}(\Omega)}
%\stopformula
%
%\noindent The set of admissible controls is denoted by ${\cal U}_{\rm ad}$.
%\stoptext
%
% ConTeXt's default body font is: modern-designsize
% as defined in: 
%
%font-set.mkvi:54:     {\def\m_font_fallback_name{modern-designsize}}%       % this might become 'modern'
%
% note that it is defined via \m_font_fallback_name as part of the default
% font loading process in font-set.mkvi. As such the above might not work.


\def\longRightIsoArrow{
  \setbox\diSimpTmpBox=\hbox{$\longrightarrow$}
  \diSimpTmpDimen\wd\diSimpTmpBox
  \mathrel{\hbox{\rlap{$\longrightarrow$}\raise0.7ex\hbox to \diSimpTmpDimen {\hfil$\sim$\hfil\hfil}}}
}

\def\joylol{\inlinemath{\type{JoyLoL}}}
\def\joylolZero{\inlinemath{\type{JoyLoL}_0}}
\def\joylolMinus{\inlinemath{\type{JoyLoL}_{-1}}}
\def\joylolCoAlg{\inlinemath{\type{JoyLoLCoAlg}}}
\def\joylolSExp{\inlinemath{\type{JoyLoLSExp}}}
\def\joylolRM{\inlinemath{\type{JoyLoLRM}}}
\def\joylolCF{\inlinemath{\type{JoyLoLCF}}}
\def\jPeg{\inlinemath{\type{jPEG}}}
\def\jPEG{\inlinemath{\type{jPEG}}}
\def\lol{\inlinemath{\text{LoL}}}
\def\wflol{\inlinemath{\text{LoL}_{\rm wf}}}
\def\lols{\inlinemath{\text{LoLs}}}
\def\wflols{\inlinemath{\text{wf-LoLs}}}

\def\catLoL{\inlinemath{\text{\bf LoL}}}
\def\catWFLoL{\inlinemath{\text{\bf LoL}_{\rm wf}}}
\def\funcLoL{\inlinemath{L}}
\def\funcWFLoL{\inlinemath{L_{\rm wf}}}

\def\Set{\inlinemath{\text{\bf Set}}}

\def\explicitCompose#1#2{\inlinemath{#1;#2}}
\def\compose#1#2{\inlinemath{#1 \; #2}}
\def\ndChoice#1#2{\inlinemath{#1 \Vert #2}}
\def\repets#1{\inlinemath{#1^{\star}}}
\def\test#1{\inlinemath{#1?}}

\def\true{\type{true}}
\def\false{\type{false}}
\def\isNil{\type{isNil}}
\let\bottom=\bot
\let\emptySet=\emptyset
\let\infinity=\infty

\unexpanded\def\powerSet{\dosingleempty\doPowerSet}
\unexpanded\def\doPowerSet[#1]{%
  \iffirstargument\doPowerSetSingle{#1}%
  \else\doPowerSetZero%
  \fi%
}
\def\doPowerSetSingle#1{\inlinemath{\cal P(#1)}}
\def\doPowerSetZero{\inlinemath{\cal P}}

\def\sets{\inlinemath{\text{Set}}}

\startuseMPgraphic{MMundiNarrowTopBar}{textWidth,textHeight}
  draw
    (0                ,0                       ) --
    (0                ,0.75*\MPvar{textHeight}  ) --
    (\MPvar{textWidth},0.75*\MPvar{textHeight}  )
    withpen   pencircle scaled 2pt
    withcolor \MPcolor{gray} ;
\stopuseMPgraphic

\startuseMPgraphic{MMundiNarrowBottomBar}{textWidth,textHeight}
  draw 
    (0                ,\MPvar{textHeight}  ) -- 
    (\MPvar{textWidth},\MPvar{textHeight}  ) -- 
    (\MPvar{textWidth},1.75*\MPvar{textHeight})
    withpen   pencircle scaled 2pt
    withcolor  \MPcolor{gray} ;
\stopuseMPgraphic

\def\MMundi#1{%
\setbox\diSimpTmpBox=\hbox{#1}%
\diSimpTmpDimen\wd\diSimpTmpBox%
\setupMPvariables[MMundiNarrowTopBar]%
[textWidth=1.05\diSimpTmpDimen,textHeight=0.5\lineheight]%
\setupMPvariables[MMundiNarrowBottomBar]%
[textWidth=1.05\diSimpTmpDimen,textHeight=0.5\lineheight]%
\hbox to 1.05\diSimpTmpDimen {%
\rlap{\raise0.5ex\hbox{\useMPgraphic{MMundiNarrowTopBar}}}%
\rlap{\lower0.5ex\hbox{\useMPgraphic{MMundiNarrowBottomBar}}}%
\hfil\hbox{#1}\hfil}}

\startuseMPgraphic{MMundiTopBar}{textWidth,textHeight}
  draw
    (0                ,0                   ) --
    (0                ,\MPvar{textHeight}  ) --
    (\MPvar{textWidth},\MPvar{textHeight}  )
    withpen   pencircle scaled 3pt
    withcolor \MPcolor{gray} ;
    
  label("\color[darkgray]{Classical commentary}", (0.25*\MPvar{textWidth}, \MPvar{textHeight})) ;
    
  setbounds currentpicture to
    (0                ,0                   ) --
    (0                ,2*\MPvar{textHeight}) --
    (\MPvar{textWidth},2*\MPvar{textHeight}) --
    (\MPvar{textWidth},0                   ) --
    cycle ;
\stopuseMPgraphic

\startuseMPgraphic{MMundiBottomBar}{textWidth,textHeight}
  draw 
    (0                ,\MPvar{textHeight}  ) -- 
    (\MPvar{textWidth},\MPvar{textHeight}  ) -- 
    (\MPvar{textWidth},2*\MPvar{textHeight})
    withpen   pencircle scaled 3pt
    withcolor  \MPcolor{gray} ;
    
  setbounds currentpicture to
    (0                ,0                   ) --
    (0                ,2*\MPvar{textHeight}) --
    (\MPvar{textWidth},2*\MPvar{textHeight}) --
    (\MPvar{textWidth},0                   ) --
    cycle ;
\stopuseMPgraphic

\def\startMMundi{
  \setupMPvariables[MMundiTopBar]
    [textWidth=\textwidth,textHeight=0.5\lineheight]
  \useMPgraphic{MMundiTopBar}
}

\def\stopMMundi{
  \setupMPvariables[MMundiBottomBar]
    [textWidth=\textwidth,textHeight=0.5\lineheight]
  \endgraf\useMPgraphic{MMundiBottomBar}
}

\def\ToDo#1{\color[red]{\bf TODO: #1}}
\let\TODO\ToDo

\stopMkIVCode
