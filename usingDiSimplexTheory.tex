% LaTeX source for the usingDiSimplexTheory document
%

\documentclass[a4paper]{amsart}
\usepackage{perceptisys}
\usepackage{perceptisysArticle}

\begin{document}
\sloppy

\title[Using DiSimplicil Theory]{Using Directed Simplicial Type Theory 
to formally check ``informal'' mathematics}
%
\collectionTitle{A mathematical theory of Reality}
%
%% author: stg
\author{Stephen Gaito}
\address{PerceptiSys Ltd, 21 Gregory Ave, Coventry, CV3 6DJ, United Kingdom}%
\email{stephen@perceptisys.co.uk}%
\urladdr{http://www.perceptisys.co.uk}


%% version summary
\thanks{Created: 2014-05-13}
\thanks{Git commit \gitReferences{} (\gitAbbrevHash{}) commited on \gitAuthorDate{} by \gitAuthorName{}}
\thanks{AMS-\LaTeX{}'ed on \today{}.}

%% Copyrights
\thanks{\textbf{Copyright: \copyright{} Stephen Gaito, PerceptiSys Ltd \the\year{}; Some rights reserved}}
\thanks{\textbf{This work is licensed under a Creative Commons Attribution-ShareAlike 4.0 International License.}}
 
\subjclass[2010]{Primary unknown; Secondary unknown} % 
\keywords{Keyword one, keyword two etc.}%

\begin{abstract}
%
All mathematical papers should be formally checkable.  This article 
explores the challenges and solutions used to check the mathematical 
statements contained in the Mathematical theory of Reality cycle of 
papers.
%
\end{abstract}
\maketitle
\tableofcontents

\section{Introduction}

\begin{quote}
%
Alice laughed. "There's no use trying," she said: "one can't believe 
impossible things." 

"I daresay you haven't had much practice," said the Queen. 

"When I was your age, I always did it for half-an-hour a day. Why, 
sometimes I've believed as many as six impossible things before 
breakfast."

Alice and the Red Queen, \textit{Through the Looking-Glass, and What 
Alice Found There}, in Chapter 5, published 1871, by Lewis Carroll (Charles Lutwidge 
Dodgeson)

\verb|URL: http://en.wikiquote.org/wiki/Through_the_Looking-Glass|
%
\end{quote}

\subsection{First steps: Proofs from the HoTT book}

We start by looking at the first few ``formal'' proofs in the HoTT 
book.  These proofs had both an long and short form of each proof to 
provide the reader with an understanding of how much more information 
needs to be recorded to allow automatic proof checking using HoTT (for 
Coq).  For the HoTT book\footnote{See: 
\texttt{http://homotopytypetheory.org/book/}}, these proofs were checked 
in an even more formal format which was then `ìnformalized'' for use in 
the book itself.

In this article we explore how to proceed in the opposite direction: 
how ``formal'' does an informal proof need to be to be parsed by a HoTT 
based tool?

Consider:
%
\begin{verbatim}
%
We now derive from the induction principle the beginnings of the 
structure of a higher groupoid. We begin with symmetry of equality, 
which, in topological language, means that ``paths can be reversed''.

\begin{lem}\label{lem:opp}
  For every type $A$ and every $x,y:A$ there is a function
  \begin{equation*}
    (x= y)\to(y= x)
  \end{equation*}
  denoted $p\mapsto \opp{p}$, such that $\opp{\refl{x}}\jdeq\refl{x}$ 
  for each $x:A$. We call $\opp{p}$ the \define{inverse} of $p$.
  \indexdef{path!inverse}%
  \indexdef{inverse!of path}%
  \index{equality!symmetry of}%a
  \index{symmetry!of equality}%
\end{lem}

Since this is our first time stating something as a ``Lemma'' or 
``Theorem'', let us pause to consider what that means. Recall that 
propositions (statements susceptible to proof) are identified with 
types, whereas lemmas and theorems (statements that have been proven) 
are identified with \emph{inhabited} types. Thus, the statement of a 
lemma or theorem should be translated into a type, as in 
\autoref{sec:pat}, and its proof translated into an inhabitant of that 
type. According to the interpretation of the universal quantifier ``for 
every'', the type corresponding to \autoref{lem:opp} is \[ 
\prd{A:\UU}{x,y:A} (x= y)\to(y= x). \] The proof of \autoref{lem:opp} 
will consist of constructing an element of this type, i.e.\ deriving 
the judgment $f:\prd{A:\UU}{x,y:A} (x= y)\to(y= x)$ for some $f$. We 
then introduce the notation $\opp{(\blank)}$ for this element $f$, in 
which the arguments $A$, $x$, and $y$ are omitted and inferred from 
context. (As remarked in \autoref{sec:types-vs-sets}, the secondary 
statement ``$\opp{\refl{x}}\jdeq\refl{x}$ for each $x:A$'' should be 
regarded as a separate judgment.)

\begin{proof}[First proof]

  Assume given $A:\UU$, and let $D:\prd{x,y:A}(x= y) \to \type$ be the 
  type family defined by $D(x,y,p)\defeq (y= x)$. In other words, $D$ is 
  a function assigning to any $x,y:A$ and $p:x=y$ a type, namely the type 
  $y=x$.  Then we have an element
  \begin{equation*}
    d\defeq \lam{x} \refl{x}:\prd{x:A} D(x,x,\refl{x}).
  \end{equation*}
  Thus, the induction principle for identity types gives us an element
  \narrowequation{ \indid{A}(D,d,x,y,p): (y= x)}
  for each $p:(x= y)$.
  We can now define the desired function $\opp{(\blank)}$ to be 
  $\lam{p} \indid{A}(D,d,x,y,p)$, i.e.\ we set $\opp{p} \defeq 
  \indid{A}(D,d,x,y,p)$. The conversion rule~\eqref{eq:Jconv} gives 
  $\opp{\refl{x}}\jdeq \refl{x}$, as required.
\end{proof}

We have written out this proof in a very formal style, which may be 
helpful while the induction rule on identity types is unfamiliar. To be 
even more formal, we could say that \autoref{lem:opp} and its proof 
together consist of the judgment
%
\begin{narrowmultline*}
  \lam{A}{x}{y}{p} \indid{A}((\lam{x}{y}{p} (y=x)), (\lam{x} \refl{x}), x, y, p)
  \narrowbreak : \prd{A:\UU}{x,y:A} (x= y)\to(y= x)
\end{narrowmultline*}
%
(along with an additional equality judgment). However, eventually we 
prefer to use more natural language, such as in the following 
equivalent proof.

\begin{proof}[Second proof]
%
  We want to construct, for each $x,y:A$ and $p:x=y$, an element 
  $\opp{p}:y=x$.
%
  By induction, it suffices to do this in the case when $y$ is $x$ and 
  $p$ is $\refl{x}$.
%
  But in this case, the type $x=y$ of $p$ and the type $y=x$ in which 
  we are trying to construct $\opp{p}$ are both simply $x=x$.
%
  Thus, in the ``reflexivity case'', we can define $\opp{\refl{x}}$ to 
  be simply $\refl{x}$.
%
  The general case then follows by the induction principle, and the 
  conversion rule $\opp{\refl{x}}\jdeq\refl{x}$ is precisely the proof in 
  the reflexivity case that we gave.
%
\end{proof}
%
\end{verbatim}


\bibliographystyle{amsalpha}
\bibliography{usingDiSimplexTheory}

\end{document}

