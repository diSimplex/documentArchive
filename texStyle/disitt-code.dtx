% \changes{0.1b}{2015/12/22}{Initial version}
%
% \iffalse
%<*never>
\documentclass{amsart}
\usepackage{doc}
\usepackage{disitt}
\usepackage{hyperref}
\newcommand{\define}[2]{\emph{\textbf{#1}}}
\GetFileInfo{disitt-engine.sty}
\CodelineIndex
\EnableCrossrefs
\RecordChanges
\begin{document}
\CheckSum{12}
\DocInput{disitt-code.dtx}
\section{Copyright and \LaTeX\ Project Public License}
\begin{verbatim}
%txsBeginComment
Copyright (C) 2015 Stephen Gaito (PerceptiSys Ltd)

This work may be distributed and/or modified under the conditions of the
LaTeX Project Public License, either version  1.3  of  this license or (at
your option) any later version.

The latest version of this license is in
http://www.latex-project.org/lppl.txt and  version 1.3 or later is part of
all distributions of LaTeX version 2005/12/01 or later.

This work has the LPPL maintenance status `maintained'.

The Current Maintainer of this work is Stephen Gaito.

The released source can be found at:
https://github.com/diSimplex/diSimplexEngine/tree/master/texStyle  The

Development source can be found at:
https://github.com/stephengaito/diSimplexEngine/tree/master/texStyle

This work consists of the files disitt.tex, disitt-example.tex, 
disitt.dtx, disittSymbols.dtx and disitt.ins and the derived files:
disitt.sty, disitt.cwl, disittSymbols.sty, disittSymbols.cwl
and disitt.pdf.
%txsEndComment
% \end{verbatim}
\PrintIndex
\PrintChanges
\end{document}
%</never>
% \fi
%
% \DoNotIndex{\edef,\advance,\else,\csname,\endcsname,\expandafter}
% \DoNotIndex{\ifx,\fi,\input,\ifnum,\relax,\newcount,\romannumeral}
% \DoNotIndex{\the,\typeout,\xdef,\global,\newcommand,\RequirePackage}
% \DoNotIndex{\PackageError,\ProvidesPackage,\gdef}
% \DoNotIndex{\IncludeTests,\LogClose,\LogTests,\Expect,\ExpectIfThen}
% \DoNotIndex{\isnamedefined,\isundefined}
% \DoNotIndex{\stop,\makeatletter,\makeatother,\begin,\end,\def,\let}
% \DoNotIndex{\testFSDestroy@fifo@stack@count@top}
% \DoNotIndex{\testFSDestroy@fifo@stack@count@bottom}
% \DoNotIndex{\testFSDestroy@fifo@stack@count@size}
% \DoNotIndex{\fifo@stack@top@testFSDestroy}
% \DoNotIndex{\fifo@stack@bottom@testFSDestroy}
%
%
% \title{Directed Simplicial Type Theory \LaTeX\ Style}
% \author{Stephen Gaito}
% \maketitle
% \begin{abstract}
% This is the Directed Simplicial Type Theory (DiSiTT) style file. It is a LaTeX
% style file used to typeset DiSiTT proof arguments.
% \end{abstract}
%
% \tableofcontents
%
% \section{Introduction}
%
% \section{Integration tests and example usage}
%
% We use the \href{http://www.ctan.org/pkg/qstest}{QSTest} package from
% \href{http://www.ctan.org}{CTAN} to provide both integration and unit tests of
% our FIFO/stack package.
%
% In this section we walk through a number of integration tests which also
% provide example uses of FIFO/Stacks. We provide unit tests of specific
% invariants in the code section below.  These unit tests are associated
% with the section of code which implements a particular invariant.
%
% We begin by setting up the \LaTeX\ QSTest package to test the
% \verb|fifo-stack| package, and we will log everything. Note that we do not use
% a document
% class
% or begin/end a document, this is because there should not be any \emph{normal}
% output created.  All \emph{results} are listed in the associated
% \verb|fifo-stack-test.lgout| file.
%
%    \begin{macrocode}
%<*qstest>
\RequirePackage{qstest}
\RequirePackage{xifthen}
\IncludeTests{*}
\LogTests{lgout}{*}{*}
%</qstest>
%    \end{macrocode}
%
% \section{DiSiTT Code texstyle}
%
% \subsection{Identification}
%
%    \begin{macrocode}
%<*package>
\ProvidesPackage{disitt-code}%
  [2015/12/22 v1.0 DiSiTT Code]
%    \end{macrocode}
%
% \section{DiSiTT Backus-Naur Form}
%
% This \LaTeX code has been minimally altered from Adrian Robson's 
% backnaur CTAN package ( \verb|https://www.ctan.org/pkg/backnaur| ).
%
% \subsection{BNF environments}
%
% \begin{environment}{bnf}
% A production list with line numbers.
%    \begin{macrocode}
%txs\begin{bnf}#
%txs\end{bnf}#
\newenvironment{bnf}%
{\begin{eqnarray}}%
{\end{eqnarray}\ignorespacesafterend}
%    \end{macrocode}
% \end{environment}
%
% \begin{environment}{bnf*}
% A production list with line numbers.
%    \begin{macrocode}
%txs\begin{bnf*}#
%txs\end{bnf*}#
\newenvironment{bnf*}%
{\begin{eqnarray*}}%
{\end{eqnarray*}\ignorespacesafterend}
%    \end{macrocode}
% \end{environment}
%
% \subsection{BNF definition commands}
%
% \begin{macro}{\bnfPN}
%    \begin{macrocode}
%txs\bnfPN{non-terminal}#
\newcommand{\bnfPN}[1]{\langle \textnormal{#1} \rangle}
%    \end{macrocode}
% \end{macro}
%
% \begin{macro}{\bnfOR}
%    \begin{macrocode}
%txs\bnfOR#
\newcommand{\bnfOR}{\; \mid \;}
%    \end{macrocode}
% \end{macro}
%
% \begin{macro}{\bnfSP}
%    \begin{macrocode}
%txs\bnfSP#
\newcommand{\bnfSP}{\;}
%    \end{macrocode}
% \end{macro}
%
% \begin{macro}{\bnfES}
%    \begin{macrocode}
%txs\bnfES#
\newcommand{\bnfES}{\lambda}
%    \end{macrocode}
% \end{macro}
%
% \begin{macro}{\bnfTD}
%    \begin{macrocode}
%txs\bnfTD{terminal}#
\newcommand{\bnfTD}[1]{\textnormal{#1}}
%    \end{macrocode}
% \end{macro}
%
% \begin{macro}{\bnfSK}
%    \begin{macrocode}
%txs\bnfSK#
\newcommand{\bnfSK}{\dots}
%    \end{macrocode}
% \end{macro}
%
% \begin{macro}{\bnfKS}
%    \begin{macrocode}
%txs\bnfKS{base}{expoent}#
\newcommand{\bnfKS}[2]{\lbrace #1 \rbrace^{#2}}
%    \end{macrocode}
% \end{macro}
%
% \begin{macro}{\bnfPO}
%    \begin{macrocode}
%txs\bnfPO#
\newcommand{\bnfPO}{::=}
%    \end{macrocode}
% \end{macro}
%
% \begin{macro}{\bnfProd}
%    \begin{macrocode}
%txs\bnfProd{nonTerminal}{production}#
\newcommand{\bnfProd}[2]{\bnfPN{#1} & \bnfPO & #2 \\}
%    \end{macrocode}
% \end{macro}
%
% \begin{macro}{\bnfAlt}
%    \begin{macrocode}
%txs\bnfAlt{alternate production}#
\newcommand{\bnfAlt}[1]{ & \bnfOR & #1 \\}
%    \end{macrocode}
% \end{macro}
%
% \section{Code environments}
%
% \TODO{Consider using: github: mchakravarty/haskell.sty to do reserved 
% words}
%
%    \begin{macrocode}
\RequirePackage{fancyvrb}%
%    \end{macrocode}
%
% \begin{macro}{\racketRequirePkg}
%    \begin{macrocode}
%txs\racketRequirePkg{requiredPackage}#
\newcommand{\racketRequirePkg}[1]{}
%    \end{macrocode}
% \end{macro}
%
% \begin{macro}{\racketRequireFile}
%    \begin{macrocode}
%txs\racketRequireFile{requiredFileName}#
\newcommand{\racketRequireFile}[1]{}
%    \end{macrocode}
% \end{macro}
%
% \begin{environment}{racket}
%    \begin{macrocode}
%txs\begin{racket}
%txs\end{racket}
\DefineVerbatimEnvironment{racket}{Verbatim}{frame=lines, framerule=0.2mm}
%    \end{macrocode}
% \end{environment}
%
% \begin{environment}{racketPragmas}
%    \begin{macrocode}
%txs\begin{racketPragmas}
%txs\end{racketPragmas}
\newenvironment{racketPragmas}{\comment}{\endcomment}
%    \end{macrocode}
% \end{environment}
%
% \begin{environment}{haskell}
%    \begin{macrocode}
%txs\begin{haskell}
%txs\end{haskell}
\DefineVerbatimEnvironment{haskell}{Verbatim}{frame=lines, framerule=0.2mm}
%    \end{macrocode}
% \end{environment}
%
% \begin{environment}{haskellPragmas}
%    \begin{macrocode}
%txs\begin{haskellPragmas}
%txs\end{haskellPragmas}
\newenvironment{haskellPragmas}{\comment}{\endcomment}
%    \end{macrocode}
% \end{environment}
%
% \subsection{Finishing off}
%
%    \begin{macrocode}
%</package>
%<*qstest>
\LogClose{lgout}
\stop
%</qstest>
%    \end{macrocode}
%
