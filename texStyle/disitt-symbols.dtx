% \changes{0.1b}{2014/02/02}{Initial version}
%
% \iffalse
%<*never>
\documentclass{amsart}
\usepackage{doc}
\usepackage{disitt}
\usepackage{hyperref}
\newcommand{\define}[2]{\emph{\textbf{#1}}}
\GetFileInfo{disitt-engine.sty}
\CodelineIndex
\EnableCrossrefs
\RecordChanges
\begin{document}
\CheckSum{12}
\DocInput{disitt-symbols.dtx}
\section{Copyright and \LaTeX\ Project Public License}
\begin{verbatim}
%txsBeginComment
Copyright (C) 2014 Stephen Gaito (PerceptiSys Ltd)

This work may be distributed and/or modified under the conditions of the
LaTeX Project Public License, either version  1.3  of  this license or (at
your option) any later version.

The latest version of this license is in
http://www.latex-project.org/lppl.txt and  version 1.3 or later is part of
all distributions of LaTeX version 2005/12/01 or later.

This work has the LPPL maintenance status `maintained'.

The Current Maintainer of this work is Stephen Gaito.

The released source can be found at:
https://github.com/diSimplex/diSimplexEngine/tree/master/texStyle  The

Development source can be found at:
https://github.com/stephengaito/diSimplexEngine/tree/master/texStyle

This work consists of the files disitt.tex, disitt-example.tex, 
disitt.dtx, disittSymbols.dtx and disitt.ins and the derived files:
disitt.sty, disitt.cwl, disittSymbols.sty, disittSymbols.cwl
and disitt.pdf.
%txsEndComment
% \end{verbatim}
\PrintIndex
\PrintChanges
\end{document}
%</never>
% \fi
%
% \DoNotIndex{\edef,\advance,\else,\csname,\endcsname,\expandafter}
% \DoNotIndex{\ifx,\fi,\input,\ifnum,\relax,\newcount,\romannumeral}
% \DoNotIndex{\the,\typeout,\xdef,\global,\newcommand,\RequirePackage}
% \DoNotIndex{\PackageError,\ProvidesPackage,\gdef}
% \DoNotIndex{\IncludeTests,\LogClose,\LogTests,\Expect,\ExpectIfThen}
% \DoNotIndex{\isnamedefined,\isundefined}
% \DoNotIndex{\stop,\makeatletter,\makeatother,\begin,\end,\def,\let}
% \DoNotIndex{\testFSDestroy@fifo@stack@count@top}
% \DoNotIndex{\testFSDestroy@fifo@stack@count@bottom}
% \DoNotIndex{\testFSDestroy@fifo@stack@count@size}
% \DoNotIndex{\fifo@stack@top@testFSDestroy}
% \DoNotIndex{\fifo@stack@bottom@testFSDestroy}
%
%
% \title{Directed Simplicial Type Theory \LaTeX\ Style}
% \author{Stephen Gaito}
% \maketitle
% \begin{abstract}
% This is the Directed Simplicial Type Theory (DiSiTT) style file. It is a LaTeX
% style file used to typeset DiSiTT proof arguments.
% \end{abstract}
%
% \tableofcontents
%
% \section{Introduction}
%
% \section{Integration tests and example usage}
%
% We use the \href{http://www.ctan.org/pkg/qstest}{QSTest} package from
% \href{http://www.ctan.org}{CTAN} to provide both integration and unit tests of
% our FIFO/stack package.
%
% In this section we walk through a number of integration tests which also
% provide example uses of FIFO/Stacks. We provide unit tests of specific
% invariants in the code section below.  These unit tests are associated
% with the section of code which implements a particular invariant.
%
% We begin by setting up the \LaTeX\ QSTest package to test the
% \verb|fifo-stack| package, and we will log everything. Note that we do not use
% a document
% class
% or begin/end a document, this is because there should not be any \emph{normal}
% output created.  All \emph{results} are listed in the associated
% \verb|fifo-stack-test.lgout| file.
%
%    \begin{macrocode}
%<*qstest>
\RequirePackage{qstest}
\RequirePackage{xifthen}
\IncludeTests{*}
\LogTests{lgout}{*}{*}
%</qstest>
%    \end{macrocode}
%
% \section{DiSiTT Symbols texstyle}
%
% \subsection{Identification}
%
%    \begin{macrocode}
%<*package>
\ProvidesPackage{disitt-symbols}%
  [2014/03/25 v1.0 DiSiTT Symbols]
%    \end{macrocode}
%
% \subsection{Rotating Math Symbols}
%
% The symbol manipulation commands are for use with the graphicx package. They
% have been taken from an email from Barbara Beeton (bnb at ams.org). See:
% http://www.tug.org/pipermail/texhax/2005-October/004861.html. We will make
% extensive use of these commands to provide symbols for our specific use.
%
% \begin{macro}{\reflectit}
%
% The \verb|\reflectit| command takes any math symbol and reflects it in its
% vertical middle line.
%
%    \begin{macrocode}
%txs\reflectit{mathSymbol}#m
\newcommand{\reflectit}[1]{\reflectbox{\ensuremath#1}}
%    \end{macrocode}
% \end{macro}
%
% \begin{macro}{\turnover}
%
% The \verb|\turnover| command takes any math symbol and reflects it in its 
% horizontal middle line
%
%    \begin{macrocode}
%txs\turnover{mathSymbol}#m
\newcommand{\turnover}[1]{\rotatebox[origin=c]{180}{\ensuremath#1}}
%    \end{macrocode}
% \end{macro}
%
% \begin{macro}{\turncw}
%   \begin{macro}{\turnccw}
%
% The \verb|\turnXXX| commands rotate a math symbol by 90 degrees about its
% middle in either a clockwise or counter clockwise direction.
%
%    \begin{macrocode}
%txs\turncw{mathSymbol}#m
%txs\turnccw{mathSymbol}#m
\newcommand{\turncw}[1]{\rotatebox[origin=c]{270}{\ensuremath#1}}
\newcommand{\turnccw}[1]{\rotatebox[origin=c]{90}{\ensuremath#1}}
%    \end{macrocode}
%   \end{macro}
% \end{macro}

% \begin{macro}{\turnne}
%   \begin{macro}{\turnnw}
%     \begin{macro}{\turnsw}
%       \begin{macro}{\turnse}
% The \verb|\turnXX| commands rotates a math symbol by a multiple of 45
% degrees about its center to point its top in either a north east, 
% north west, south west or south east direction
%    \begin{macrocode}
%txs\turnne{mathSymbol}#m
%txs\turnnw{mathSymbol}#m
%txs\turnsw{mathSymbol}#m
%txs\turnse{mathSymbol}#m
\newcommand{\turnne}[1]{\rotatebox[origin=c]{45}{\ensuremath#1}}
\newcommand{\turnnw}[1]{\rotatebox[origin=c]{135}{\ensuremath#1}}
\newcommand{\turnsw}[1]{\rotatebox[origin=c]{225}{\ensuremath#1}}
\newcommand{\turnse}[1]{\rotatebox[origin=c]{315}{\ensuremath#1}}
%    \end{macrocode}
%       \end{macro}
%     \end{macro}
%   \end{macro}
% \end{macro}
%
% \subsection{Editorial comments}
%
% \begin{macro}{\CHECK}
%   \begin{macro}{\COMMENT}
%     \begin{macro}{\TODO}
% The editorial commands \verb|\CHECK|, \verb|\COMMENT| and \verb|\TODO| are
% provided to allow for editorial comments to the author but which will not be
% included in the final document.
%    \begin{macrocode}
%txs\CHECK{someThingToCheck}#n
\newcommand{\CHECK}[1]{\textbf{CHECK:[} #1 \textbf{]}}
%txs\COMMENT{anEditorialComment}#n
\newcommand{\COMMENT}[1]{\textbf{COMMENT:[} #1 \textbf{]}}
%txs\TODO{somethingThatNeedsToBeDone}#n
\newcommand{\TODO}[1]{\textbf{TODO: [} \ifmmode\text{#1}\else#1\fi \textbf{]}}
%    \end{macrocode}
%     \end{macro}
%   \end{macro}
% \end{macro}
%
% \subsection{Defininig a word or phrase}
%
% \begin{macro}{\define}
% The \verb|\define| command ensures that the defined word is listed in the
% index as well as placed in emphasized text font
%    \begin{macrocode}
%txs\define{word}{something...}#n
\newcommand{\define}[2]{\emph{\textbf{#1}}}
%    \end{macrocode}
% \end{macro}
%
% \subsection{Book keeping}
%
% We define the following ``book keeping'' commands:
%
% \begin{macro}{\variable}
%    \begin{macrocode}
%txs\variable{aVarSym}{aSubScript}#
\newcommand{\variable}[2]{\ensuremath{#1_{#2}}}
%    \end{macrocode}
% \end{macro}
%
%\subsection{Important Objects}
%\subsubsection{Universe}
%
% We define Plato's Universe:
%
% \begin{macro}{\Universe}
%    \begin{macrocode}
%txs\Universe#
\newcommand{\Universe}{\ensuremath{\mathfrak{U}}}
%    \end{macrocode}
% \end{macro}
%
%\subsubsection{Lists}
%
% We define ordered collections known by computer scientists as \emph{Lists}:
%
% \begin{macro}{\Lists}
%    \begin{macrocode}
%txs\Lists#
\newcommand{\Lists}{\ensuremath{\mathfrak{L}}}
%    \end{macrocode}
% \end{macro}
%
%\subsubsection{List Automorphisms}
%
% We define the automorphisms of Lists:
%
% \begin{macro}{\ListAutomorphisms}
%    \begin{macrocode}
%txs\ListAutomorphisms#
\newcommand{\ListAutomorphisms}{\ensuremath{\Delta}}
%    \end{macrocode}
% \end{macro}
%
%\subsubsection{Ordinals}
% We define the following standard objects:
%
% \begin{macro}{\Ordinal}
%   \begin{macro}{\zero}
%     \begin{macro}{\one}
%    \begin{macrocode}
%txs\Ordinal#
\newcommand{\Ordinal}{\ensuremath{\mathcal{O}}}
%txs\zero#
\newcommand{\zero}{\ensuremath{0}}
%txs\one#
\newcommand{\one}{\ensuremath{1}}
%    \end{macrocode}
%     \end{macro}
%   \end{macro}
% \end{macro}
%
% Now some ``functions'':
%
% \begin{macro}{\successor}
%   \begin{macro}{\limitOrd}
%    \begin{macrocode}
%txs\successor{anOrdinal}#
\newcommand{\successor}[1]{\ensuremath{s(#1)}}
%txs\limitOrd{anOrdinal}#
\newcommand{\limitOrd}[1]{\ensuremath{\omega(#1)}}
%    \end{macrocode}
%   \end{macro}
% \end{macro}
%
% \subsection{Deltas}
%
% \begin{macro}{\diSimplex}
%    \begin{macrocode}
%txs\diSimplex{commaListDiSimplicies}#
\newcommand{\diSimplex}[2]{\ensuremath{\Delta_{#2} \langle #1 \rangle}}
%    \end{macrocode}
% \end{macro}
%
% \subsection{Important Arrows}
%
% \begin{macro}{\arrow}
%    \begin{macrocode}
%txs\arrow{domainObject}{targetObject}#
\newcommand{\arrow}[2]{\ensuremath{#1 \rightarrow #2}}
%    \end{macrocode}
% \end{macro}
%
% \subsection{Cardinals}
%
% \begin{macro}{\Cardinal}
%    \begin{macrocode}
%txs\Cardinal#
\newcommand{\Cardinal}{\ensuremath{\mathcal{C}}}
%    \end{macrocode}
% \end{macro}
%
% \subsection{Sets}
%
% \begin{macro}{\Set}
%    \begin{macrocode}
%txs\Set#
\newcommand{\Set}{\ensuremath{\textbf{Set}}}
%    \end{macrocode}
% \end{macro}
%
% \subsection{Isomorphisms}
%
% \begin{macro}{\isomorphic}
%    \begin{macrocode}
%txs\isomorphic#
\newcommand{\isomorphic}{\ensuremath{\backsimeq}}
%    \end{macrocode}
% \end{macro}
%
% \subsection{Finishing off}
%
%    \begin{macrocode}
%</package>
%<*qstest>
\LogClose{lgout}
\stop
%</qstest>
%    \end{macrocode}
%
