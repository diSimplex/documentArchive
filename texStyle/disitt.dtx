% \changes{0.1b}{2014/02/02}{Initial version}
%
% \iffalse
%<*never>
\documentclass{amsart}
\usepackage{doc}
\usepackage{disitt}
\usepackage{hyperref}
\newcommand{\define}[2]{\emph{\textbf{#1}}}
\GetFileInfo{disitt-engine.sty}
\CodelineIndex
\EnableCrossrefs
\RecordChanges
\begin{document}
\CheckSum{12}
\DocInput{disitt.dtx}
%\DocInput{disitt-symbols.dtx}
\section{Copyright and \LaTeX\ Project Public License}
\begin{verbatim}
%txsBeginComment
Copyright (C) 2014 Stephen Gaito (PerceptiSys Ltd)

This work may be distributed and/or modified under the conditions of the
LaTeX Project Public License, either version  1.3  of  this license or (at
your option) any later version.

The latest version of this license is in
http://www.latex-project.org/lppl.txt and  version 1.3 or later is part of
all distributions of LaTeX version 2005/12/01 or later.

This work has the LPPL maintenance status `maintained'.

The Current Maintainer of this work is Stephen Gaito.

The released source can be found at:
https://github.com/diSimplex/diSimplexEngine/tree/master/texStyle  The

Development source can be found at:
https://github.com/stephengaito/diSimplexEngine/tree/master/texStyle

This work consists of the files disitt.tex, disitt-example.tex, 
disitt.dtx, disitt-symbols.dtx and disitt.ins and the derived files:
disitt.sty, disitt.cwl, disitt-symbols.sty, disitt-symbols.cwl
and disitt.pdf.
%txsEndComment
% \end{verbatim}
\PrintIndex
\PrintChanges
\end{document}
%</never>
% \fi
%
% \DoNotIndex{\edef,\advance,\else,\csname,\endcsname,\expandafter}
% \DoNotIndex{\ifx,\fi,\input,\ifnum,\relax,\newcount,\romannumeral}
% \DoNotIndex{\the,\typeout,\xdef,\global,\newcommand,\RequirePackage}
% \DoNotIndex{\PackageError,\ProvidesPackage,\gdef}
% \DoNotIndex{\IncludeTests,\LogClose,\LogTests,\Expect,\ExpectIfThen}
% \DoNotIndex{\isnamedefined,\isundefined}
% \DoNotIndex{\stop,\makeatletter,\makeatother,\begin,\end,\def,\let}
% \DoNotIndex{\testFSDestroy@fifo@stack@count@top}
% \DoNotIndex{\testFSDestroy@fifo@stack@count@bottom}
% \DoNotIndex{\testFSDestroy@fifo@stack@count@size}
% \DoNotIndex{\fifo@stack@top@testFSDestroy}
% \DoNotIndex{\fifo@stack@bottom@testFSDestroy}
%
% \title{Directed Simplicial Type Theory \LaTeX\ Style}
% \author{Stephen Gaito}
% \maketitle
% \begin{abstract}
% This is the Directed Simplicial Type Theory (DiSiTT) style file. It is a LaTeX
% style file used to typeset DiSiTT proof arguments.
% \end{abstract}
%
% \tableofcontents
%
% \section{Introduction}
%
% The Directed Simplicial Type Theory (DiSiTT) style is a \LaTeX\ style designed
% to type set DiSiTT style natural decuction derivations, either rules or chains
% of rule derivations. The \emph{primary purpose} of using the DiSiTT style
% instead of other \LaTeX\ natural deduction packages is that the DiSiTT style
% provides the syntax required by the DiSiTT engine to perform automatic
% derivation checking. \emph{Inside} \LaTeX\ the DiSiTT style provides a
% unified syntax with in which either the ``Gentzen'' (tree) style (using
% \verb|bussproofs.sty|) or ``Fitch'' (sequential) style (using \verb|fitch.sty|) can be
% used.
%
% The Directed Simplicial Type Theory (DiSiTT) engine, is essentially a natural
% deduction judgement programming language (JPL). In classical natural deduction
% the basic \emph{judement} is a statement that a given proposition, $A$, is
% \emph{true}. In DiSiTT, the basic judgements are statements that one
% (directed) simplicial structure is a \emph{part} of another.
%
% Any programming language will need to keep track of variables in various
% contexts, our JPL is no different.  Natural deductions consist of nested
% collections of derivations or \emph{proofs}.  Variables are scoped to a
% particular nested collection of derivations.
%
% We have a number of distinct types of derivations:
%
% \begin{enumerate}
%   \item \define{Definition}{}\DescribeEnv{def}\ or
%     \define{Subdefinition}{}\DescribeEnv{subdef}\ used to \emph{define} an
%     equivilance between two statements. Subdefinitions are numbered relative to
%     their enclosing (sub)definition(s).
%   \item \define{Lemma}{} or \define{Theorem}{} used to provide a number of
%     \emph{hypothosized} statements followed by a number of \emph{derived}
%     statements. A Lemma or Theorem is a definition of a natural deduction rule
%     promise that a Proof will be given either in the following text, an
%     appendix, or from another work.
%   \item \define{Conjecture}{} the same as a Lemma or Theorem statement but
%     marked as unproven (and hence unuseable in other derivations).
%   \item \define{Proof}{} or \define{Subproof}{} used to provide detailed
%     natural deduction derivations of one or more \emph{derived} statements using
%     the given \emph{hypothosized} statements. Typically a proof follows a given
%     Lemma or Theorem.  However Proofs can be used as a running discusion on how
%     to prove a given statement (which might only be visible by the end of the
%     discussion).
% \end{enumerate}
%
% One of our primary aims is to produce ``journal'' ready documents, for this we
% use \LaTeX\ as our primary document environment and assume that the JPL
% derivations are environments inside a \LaTeX\ document. As such each of the
% above derivation types are defined as \LaTeX\ environments, for example:
%
% \begin{verbatim}
%   \begin{definition}
%     \begin{subdefinition}
%     ....
%     \end{subdefinition}
%   \end{definition}
% \end{verbatim}
%
% Each derivation environment consists of a number of derivation statements. A
% single statement consists of an optional \define{context}{} followed by a
% \define{judgement}{}, where a context is a sequence of judgements:
%
% \begin{verbatim}
%   \hypothesis{
%     \judgement{a}{A}, 
%     \judgement{b}{B},
%      ...
%     \judgement{x}{X}
%   }{
%     \judgement{z}{Z}
%   }
% \end{verbatim}
% or
% \begin{verbatim}
%   \conclusion{
%     \judgement{a}{A}, 
%     \judgement{b}{B},
%      ...
%     \judgement{x}{X}
%   }{
%     \judgement{z}{Z}
%   }
% \end{verbatim}
%
% Every derivation environment must have at least one \verb|\conclusion|
% statement. Multiple \verb|\conclusion| statements are considered to be mutulally
% equivilant. Derivation environments need not have any \verb|\hyposthesis|
% statements.  The order of statements is significant, and acts as a stack.
%
% In a proof derivation, every \verb|\conclusion| statement must be followed by a
% \verb|\followsUsing{xxx}| statement, which identifies the previously defined
% derivation which justifies the previous conclusion.
%
% \section{Examples}
%
% In this section we provide some worked examples.
% \begin{verbatim}
% \begin{lemma}
%   The following are equivilant:
%   \conclusion{}{\judgement{a}{A}}
%   \conclusion{}{\judgement{b}{B}}
% \end{lemma}
% \end{verbatim}
%
% \section{Integration tests and example usage}
%
% We use the \href{http://www.ctan.org/pkg/qstest}{QSTest} package from
% \href{http://www.ctan.org}{CTAN} to provide both integration and unit tests of
% our FIFO/stack package.
%
% In this section we walk through a number of integration tests which also
% provide example uses of FIFO/Stacks. We provide unit tests of specific
% invariants in the code section below.  These unit tests are associated
% with the section of code which implements a particular invariant.
%
% We begin by setting up the \LaTeX\ QSTest package to test the
% \verb|fifo-stack| package, and we will log everything. Note that we do not use
% a document
% class
% or begin/end a document, this is because there should not be any \emph{normal}
% output created.  All \emph{results} are listed in the associated
% \verb|fifo-stack-test.lgout| file.
%
%    \begin{macrocode}
%<*qstest>
\RequirePackage{qstest}
\RequirePackage{xifthen}
\IncludeTests{*}
\LogTests{lgout}{*}{*}
\RequirePackage{disitt-engine}
%</qstest>
%    \end{macrocode}
%
% \section{DiSiTT texstyle}
%
% \subsection{Identification}
%
%    \begin{macrocode}
%<*package>
\ProvidesPackage{disitt}%
  [2014/03/25 v1.0 DiSiTT natural deduction styles]
%txs#include:disitt-engine
\RequirePackage{disitt-engine}
%    \end{macrocode}
%
% \subsection{DiSiTT options}
%
% \subsubsection{Options for using includes}
% 
% \begin{macro}{\disittIncludeOnly}
%   \begin{macro}{includeAll}
%    \begin{macrocode}
%txs\disittIncludeOnly{aFileToPotentiallyInclude}#n
\DeclareOption{includeAll}{%
\newcommand{\disittIncludeOnly}[1]{}%
}
%    \end{macrocode}
%   \end{macro}
%
%   \begin{macro}{includeSome}
%    \begin{macrocode}
\DeclareOption{includeSome}{%
\newcommand{\disittIncludeOnly}[1]{\includeonly{##1}}%
}
%    \end{macrocode}
%   \end{macro}
% \end{macro}
%
% \subsubsection{Opions for using the TikZ diagram tool}
%
% \begin{environment}{cTikzPicture}
%   \begin{macro}{tikzAll}
%    \begin{macrocode}
%txs\begin{cTikzPicture}#n
%txs\end{cTikzPicture}#n
\DeclareOption{tikzAll}{%
\newenvironment{cTikzPicture}%
{\begin{center}\begin{tikzpicture}}%
{\end{tikzpicture}\end{center}}
}
%    \end{macrocode}
%   \end{macro}
%
%   \begin{macro}{tikzSome}
%    \begin{macrocode}
\DeclareOption{tikzSome}{%
\newenvironment{cTikzPicture}%
{\center\textbf{TikzPicture}\endcenter\comment}%
{\endcomment}%
}
%    \end{macrocode}
%   \end{macro}
% \end{environment}
%
% \subsubsection{Execution of default options}
%
%    \begin{macrocode}
\ExecuteOptions{includeSome,tikzSome}
\ProcessOptions
%    \end{macrocode}
%
% \subsection{Setting up required packages}
%
%    \begin{macrocode}
\RequirePackage{stmaryrd}%
\RequirePackage{amsmath}%
\RequirePackage{amsfonts}%
\RequirePackage{amssymb}%
\RequirePackage{cmll}%
\RequirePackage{graphicx}%
%\RequirePackage{diagxy}%
%\RequirePackage{fancyhdr}%
\RequirePackage{gitinfo}%
\RequirePackage{hyperref}% This may need to be last!
%\RequirePackage{oldgerm}% for textfrak fonts
%\RequirePackage{yfonts}% for textfrak fonts
\RequirePackage{lscape} % for landscape of individual pages (large tables)
\RequirePackage{verbatim}%
%    \end{macrocode}
%
% \subsection{Setting up our use of PGF/TikZ}
%
%    \begin{macrocode}
\usepackage{tikz}
\usetikzlibrary{calc,intersections,matrix,arrows}
%
\tikzset{comDiagM/.style={matrix of math nodes, row sep=3em, column sep=3em, text height=1.5ex, text depth=0.25ex}}
\tikzset{comDiagP/.style={->,font=\scriptsize}}
\tikzset{inFront/.style={fill=white,inner sep=2pt}}
\tikzset{injection/.style={right hook->}}
\tikzset{surjection/.style={->>}}
\tikzset{map/.style={->}}
%    \end{macrocode}
%
% \begin{environment}{cTikzPictureWorking}
%    \begin{macrocode}
%txs\begin{cTikzPictureWorking}#n
%txs\end{cTikzPictureWorking}#n
\newenvironment{cTikzPictureWorking}
{\begin{center}\begin{tikzpicture}}%
{\end{tikzpicture}\end{center}}
%    \end{macrocode}
% \end{environment}
%
% \subsection{Embedding document information in to output PDF file}
%
% We could use something like the following to embed version control info
% into the information section of the PDF file... 
% this was taken from \cite[Section 3.4.4]{scharrer2007svnLatex}
%
% \begin{verbatim}
% \pdfinfo { /CreationDate (D:\gitAuthorDate) }
% \makeatletter
% \hypersetup{%
%    pdfauthor={\gitAuthorName},
%    pdfkeywords={Revision \gitAbbrevHash}
% }
% \makeatother
% \end{verbatim}
%
%\subsection{Setting up the theorem like environments}
%
%    \begin{macrocode}
\newtheoremstyle{disittPlain}% name
	{6pt}% space above
	{6pt}% space below
	{\itshape}%    body font
	{}%    indent amount
	{\bfseries}% theorem head font
	{.}%   punctuation after theorem head
	{.5em}% space after theorem head
	{}%    theorem head spec
%
\newtheoremstyle{disittDefinition}% name
	{6pt}% space above
	{6pt}% space below
	{}%    body font
	{}%    indent amount
	{\bfseries}% theorem head font
	{.}%   punctuation after theorem head
	{.5em}% space after theorem head
	{}%    theorem head spec
%
\newtheoremstyle{disittRemark}% name
	{6pt}% space above
	{6pt}% space below
	{}%    body font
	{}%    indent amount
	{\bfseries}% theorem head font
	{.}%   punctuation after theorem head
	{.5em}% space after theorem head
	{}%    theorem head spec
%
\newtheoremstyle{disittAsside}% name
	{6pt}% space above
	{6pt}% space below
	{\itshape\small}%    body font
	{}%    indent amount
	{\bfseries}% theorem head font
	{.}%   punctuation after theorem head
	{.5em}% space after theorem head
	{}%    theorem head spec
%
\theoremstyle{disittPlain}
%
%txs\begin{conjecture}#n
%txs\end{conjecture}#n
\newtheorem{conjecture}{Conjecture}
%
%txs\begin{corollary}#n
%txs\end{corollary}#n
\newtheorem{corollary}{Corollary}[section]
%
%txs\begin{lemma}#n
%txs\end{lemma}#n
\newtheorem{lemma}{Lemma}[section]
%
%txs\begin{theorem}#n
%txs\end{theorem}#n
\newtheorem{theorem}{Theorem}[section]
%
\theoremstyle{disittDefinition}
%
%txs\begin{definition}#n
%txs\end{definition}#n
\newtheorem{definition}{Definition}[section]
\let\oldEndDefinition\enddefinition
\renewcommand{\enddefinition}{\hfill\ensuremath{\Box}\oldEndDefinition}
%
%txs\begin{example}#n
%txs\end{example}#n
\newtheorem{example}{Example}[section]
%
\theoremstyle{disittRemark}
%
%txs\begin{remark}#n
%txs\end{remark}#n
\newtheorem{remark}{Remark}[section]
%
%\theoremstyle{disittAsside}
%\newtheorem{comment}{Comment}
%
\numberwithin{equation}{section}
%    \end{macrocode}
%
% \subsection{Additional setup}
%    \begin{macrocode}
%\newcommand{\discussionIcon}{{\fontencoding{T1}\fontfamily{fsyp}\selectfont A}}
%
%\newcommand{\sectionQuote}[1]{\protect\begin{flushright}\protect\parbox{10cm}{\protect\begin{verse}#1\protect\end{verse}}\protect\end{flushright}}
%
%\RequirePackage{array}
%\setlength{\extrarowheight}{1pt}
%\renewcommand{\arraystretch}{1.25}
%\newcolumntype{C}{>{$}c<{$}}
%\newcolumntype{L}{>{$}l<{$}}
%\newcolumntype{R}{>{$}r<{$}}
%
%\DeclareGraphicsRule{*}{mps}{*}{}
%    \end{macrocode}
%
% \subsection{Pull in DiSiTT symbols}
%
%    \begin{macrocode}
%txs#include:disitt-symbols
%\usepackage{disitt-symbols}
%    \end{macrocode}
%
% \subsection{Finishing off}
%
%    \begin{macrocode}
%</package>
%<*qstest>
\LogClose{lgout}
\stop
%</qstest>
%    \end{macrocode}
%
