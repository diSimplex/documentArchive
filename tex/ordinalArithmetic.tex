% LaTeX source for the ordinalArithmetic document
%

\InputIfFileExists{options}{}{}
\documentclass[a4paper]{amsart}
\usepackage{disitt}
\usepackage{disitt-symbols}

\begin{document}
\sloppy

\title[Ordinal Arithmetic]{Ordinal arithmetic in Directed Simplical Type Theory}
%% author: stg
\author{Stephen Gaito}
\address{PerceptiSys Ltd, 21 Gregory Ave, Coventry, CV3 6DJ, United Kingdom}%
\email{stephen@perceptisys.co.uk}%
\urladdr{http://www.perceptisys.co.uk}


%% version summary
\thanks{Created: 2014-05-13}
\thanks{Git commit \gitReferences{} (\gitAbbrevHash{}) commited on \gitAuthorDate{} by \gitAuthorName{}}
\thanks{AMS-\LaTeX{}'ed on \today{}.}

%% Copyrights
\thanks{\textbf{Copyright: \copyright{} Stephen Gaito, PerceptiSys Ltd \the\year{}; Some rights reserved}}
\thanks{\textbf{This work is licensed under a Creative Commons Attribution-ShareAlike 4.0 International License.}}

\subjclass[2010]{Primary unknown; Secondary unknown} %
\keywords{Keyword one, keyword two etc.}%

\begin{abstract}
As a first basic use of Directed Simplicial Type Theory, we study Ordinal Arithmetic, starting with the definitions of the Ordinals, Ordinal Integers, Ordinal Rationals, and finally the (imprecise) Ordinal Reals and the associated (imprecise) Ordinal Complex numbers
\end{abstract}
\maketitle
\tableofcontents

\section{Introduction}

Essentially, the only and ultimate goal of Mathematics, is to build
\emph{Plato's Playground}. That is the ``collection'' of DiSiTT structures which
it is possible to build. Of great use to the Mathematical Sciences, is our
collective \emph{interpretation} of a given structure in Plato's Playground.

In order to ``understand'' our mathematics at the meta-theory level, we need to
start exploring Plato's Playground by building the Ordinals, \Ordinal.

We begin at the \emph{theory} level by positing the existence of a \emph{type},
the Ordinals, \Ordinal\ and \emph{an} object, \judgement{\zero}{\Ordinal}.

\begin{deAxiom}
\conclusion{}{\judgement{\Ordinal}{\Universe}}
\end{deAxiom}

\begin{deAxiom}
\conclusion{}{\judgement{\zero}{\Ordinal_{\cdot}}}
\end{deAxiom}

Given an ordinal, $\alpha$, we can create a new ordinal in two distinctly
different ways. We can form the \emph{successor}, \successor{\alpha}:

\begin{deAxiom}
\hypothesis{\Gamma}{\judgement{\alpha}{\Ordinal}}
\conclusion{\Gamma}{\judgement{\successor{\alpha}}{\Ordinal}}
\end{deAxiom}

We can also form a new \emph{limit} ordinal, \limitOrd{\alpha}:

\begin{deAxiom}
\hypothesis{\Gamma}{\judgement{\alpha}{\Ordinal}}
\conclusion{\Gamma}{\judgement{\limitOrd{\alpha}}{\Ordinal}}
\end{deAxiom}

So far we have posited the existence of the \emph{objects}, or \zero-simplices,
in the Ordinal structure. The quintiessential property of the Ordinals, is that
they form a totally ordered structure. To express this aspect of the Ordinals,
we need to specify the ``arrows'', or \one-simplices, in the Ordinal structure.

\begin{deAxiom}
\hypothesis{\Gamma}{\judgement{\alpha}{\Ordinal_{\cdot}}}
\conclusion{\Gamma}{
  \judgement{\arrow{\alpha}{\successor{\alpha}}}{\Ordinal_{\rightarrow}}
}
\end{deAxiom}

\begin{deAxiom}
\conclusion{}{
  \judgement{\arrow{\zero}{\limitOrd{\zero}}}{\arrow{\Ordinal}{\Ordinal}}
}
\end{deAxiom}

\begin{deAxiom}
\hypothesis{\Gamma}{\judgement{\alpha}{\Ordinal_{\cdot}}}
\conclusion{\Gamma}{
  \judgement{\arrow{\limitOrd{\alpha}}{\limitOrd{\successor{\alpha}}}}{\Ordinal_{\rightarrow}}
}
\end{deAxiom}

\begin{deAxiom}
\hypothesis{\Gamma}{\judgement{\alpha}{\Ordinal_{\cdot}}}
\hypothesis{\Gamma}{\judgement{\beta}{\Ordinal_{\cdot}}}
\hypothesis{\Gamma}{\judgement{\arrow{\alpha}{\successor{\alpha}}}{\Ordinal_{\rightarrow}}}
\hypothesis{\Gamma}{\judgement{\arrow{\alpha}{\limitOrd{\beta}}}{\Ordinal_{\rightarrow}}}
\conclusion{\Gamma}{
  \judgement{\arrow{\successor{\alpha}}{\limitOrd{\beta}}}{\Ordinal_{\rightarrow}}
}
\end{deAxiom}

\TODO{There is an implicit assumption here that the first four axioms only deal
with objects.  That is you can not form a successor or limit-ordinal from an
arrow!! How do we make this assumption explicit? }

With the very minimal tools available to us, we would now like to show that
every ordinal has a one-simplex (``arrow'') directed to the ``next''
limit-ordinal.

%\begin{deDefinition}
\begin{deAxiom}
\hypothesis{\Gamma}{\judgement{\alpha}{\Ordinal_{\cdot}}}
\conclusion{\Gamma}{
  \judgement{\arrow{\limitOrd{\alpha}}{\limitOrd{\successor{\alpha}}}}{\Ordinal_{\rightarrow}}
}
\end{deAxiom}
%\end{deDefinition}

%\begin{deDefinition}
\begin{deAxiom}
\hypothesis{\Gamma}{\judgement{\alpha}{\Ordinal_{\cdot}}}
\hypothesis{\Gamma}{\judgement{\beta}{\Ordinal_{\cdot}}}
\hypothesis{\Gamma}{\judgement{\arrow{\alpha}{\successor{\alpha}}}{\Ordinal_{\rightarrow}}}
\hypothesis{\Gamma}{\judgement{\arrow{\alpha}{\limitOrd{\beta}}}{\Ordinal_{\rightarrow}}}
\conclusion{\Gamma}{
  \judgement{\arrow{\successor{\alpha}}{\limitOrd{\beta}}}{\Ordinal_{\rightarrow}}
}
\end{deAxiom}
%\end{deDefinition}

Aren't the above two ``definitions'' really axioms and wouldn't we have a
stronger axiom if we could say that \judgement{ $
\arrow{\alpha}{\limitOrd{\beta}} =
\arrow{\arrow{\alpha}{\successor{\alpha}}}{\limitOrd{\beta}} $ }{
\Ordinal_{\rightarrow} } which is expressing the ``fact'' that there is only
\emph{one} arrow between any two objects. Alas to do this we need to understand
``identity'' of objects(?) and arrows.

Finally we have the axiom \COMMENT{is this really an axiom anymore? do we need it?}:
\newcommand{\nextLimitOrd}[1]{\ensuremath{\hat{\omega}(#1)}}
\begin{deAxiom}
\hypothesis{\Gamma}{\judgement{\alpha}{\Ordinal}}
\conclusion{\Gamma}{
  \judgement{\arrow{\alpha}{\nextLimitOrd{\alpha}}}{\Ordinal}
}
\end{deAxiom}

\TODO{Here be dragons! We could posit ``the concatination of two arrows is an
arrow'', but then how do we distinguish between ``atomic'' arrows and
``non-atomic'' (derived) arrows? Also to prove that every ordinal is an ultimate
successor of an identifible limit-ordinal, we need to essentially run the above
axioms in reverse (to provide a ``negative'' conclusion -- something can't
happen). We certainly need to introduce (co)induction!! Also, we really would
like $\limitOrd{\zero} = \zero$, how do we ``cleanly'' do this?}

\bibliographystyle{amsalpha}
\bibliography{ordinalArithmetic}

\end{document}

