% LaTeX source for the diSimplexTheory document
%

\documentclass[a4paper,openany]{amsbook}
\usepackage{disitt}
\usepackage{disitt-symbols}

\begin{document}
\frontmatter
\sloppy

\title[DiSimplicial Theory]{Reality of Mathematics: Directed Simplicial Theory}
%% author: stg
\author{Stephen Gaito}
\address{PerceptiSys Ltd, 21 Gregory Ave, Coventry, CV3 6DJ, United Kingdom}%
\email{stephen@perceptisys.co.uk}%
\urladdr{http://www.perceptisys.co.uk}


%% version summary
\thanks{Created: 2014-05-13}
\thanks{Git commit \gitReferences{} (\gitAbbrevHash{}) commited on \gitAuthorDate{} by \gitAuthorName{}}
\thanks{AMS-\LaTeX{}'ed on \today{}.}

%% Copyrights
\thanks{\textbf{Copyright: \copyright{} Stephen Gaito, PerceptiSys Ltd \the\year{}; Some rights reserved}}
\thanks{\textbf{This work is licensed under a Creative Commons Attribution-ShareAlike 4.0 International License.}}

\subjclass[2010]{Primary unknown; Secondary unknown} %
\keywords{Keyword one, keyword two etc.}%

\begin{abstract}
To be able to provide a mathematical theory of Reality, one must first 
address the question: \textit{How ``Real'' is Mathematics?}
\end{abstract} 
\maketitle 
\tableofcontents 
\mainmatter

\begin{quotation}
In the places I go there are things that I see\\
That I \emph{never} could spell if I stopped with the Z.\\
\begin{quote}
\textit{On Beyond Zebra} by Dr Seuss.
\end{quote}
\end{quotation}

\section{Introduction}

The aim of this cycle of papers is to provide a rigorous mathematical theory of
Reality. This first paper provides \emph{an} answer to the question: \emph{How
``Real'' is Mathematics?}, that is, \emph{what is the validity of the
mathematics used to provide a mathematical theory of Reality?}

Any answer to this question, which forms part of this cycle of papers, must of
necessity be circular: what is the Reality of the mathematics which provides a
mathematical description of Reality? How Real is mathematics? If it is not
Real/valid, then is Reality itself Real/valid?

To break this circularity, we take the pragmatic approach, bastardising
Descartes, and state: \emph{I make marks in Reality, so I exist}.

Any discussion of the `Reality' or `Validity' of Mathematics is essentially a
discussion of the \textit{Foundations of Mathematics}. Classically Mathematics
is founded upon Zermelo-–Fraenkel set theory with the axiom of choice (ZFC)
expressed in the language of First Order Predicate Logic. To avoid Russel's
paradox, one form of the Liar Paradox, ZFC explicitly includes an axiom forcing
all sets to be well--founded. From a Categorical point of view, this means that
classical Set Theory is an \emph{Algebraic} theory. From a computational point
of view, this means that classical Set Theory is `data'.

One of the primary aims of this cycle of papers, is to show that the Sciences
and Engineering, are better served by a Mathematics which focuses on spatially
distributed interacting \emph{processes}. This means that, from a Categorical
point of view, we are interested in \emph{Co-Algebraic} structures as the
Foundations of Mathematics.

Equally importantly, `Truth', as expressed by Aristotelian Logic, in the form of
First Order Predicate Logic, is less important then the ability to compute. We
will show that First Order Predicate Logic, as traditionally used in
Mathematics, is the \emph{internal} logic of \emph{the} Co-Algebraic Universe of
Mathematics. We will refer to this Co-Algebraic Universe as \emph{`Plato's
Universe'}, \Universe.

We base our foundations of mathematics on computation \emph{rather than} Logic.

Clearly, not all of classical Mathematics is `computable'.  For example, the use
of the (the unrestricted) Axiom of Choice, is un-computable by a classical
Turing Machine. This suggests that, to capture the whole of classical
Mathematics, we will need a more nuanced model of computation. The critical
point here is that classical mathematics is developed from an \emph{omnipotent}
point of view, yet as finite beings, we only have far more limited computational
abilities. In out best 'school Latin`, we christen beings with these more
limited abilities, as \emph{modipotent}.

The \emph{sub}universe of mathematics computable by an $\omega$-modipotent
being, we will refer to as \emph{`Plato's Playground'}.

\TODO{rework the following}

As will become obvious later in this cycle of papers, ``Reality'' is \emph{a}
model of \emph{a} higher topos.  Unfortunately, for finite beings \emph{inside}
Reality, we can only ever specify a possible collection of possible Realities. A
collection of Realities which are consistent with our current, finitely limited,
understanding, or measurement of ``Reality''.

Far more fundamental and far more important than propositions is the concept of
Causality.  I will argue, in later work, that the human brain, is a causality
inference engine. I will equally argue that the unification of Relativity and
Quantum Mechanics requires a careful understanding of the \emph{finite}
structure of causality.

As should become evident, developing the mathematics of the \emph{finite}
structure of causality, is most easily done when based upon Directed Simplicial
Theory rather than classical propositional/predicate calculus. Following the
lead of HoTT, propositional/predicate calculus, as well as set theory will be
specific ``universes'' of Directed Simplicial Type Theory.

The important point here, is that, while the mathematics of causality is
\emph{implicitly} ``part'' of classical logic, it is conceptually useful to make
the structure of causality \emph{explicit} in our foundations of mathematics. 
Inferring and manipulating causality is after all the main aim of all of the
sciences and engineering.

This book is devoted to giving the reader a more ``mathematical'' overview of
Directed Simplicial Theory.  It is meant to be read in companion with the very
much more detailed \textit{Directed Simplicial Theory Implemented in Haskell}.

\section{Building Plato's Wilderness}

Our explicit thesis is that Mathematics \emph{is} computation.

To begin, we must provide a model of computation. The classical models of
computability, such as Turing Machines, the Lambda Calculus and General
Recursive Functions, are all, more or less, about computational sequentiallity
and `time'. While the model we are about to propose will be equivilant to any of
the classical models, it is designed to be \emph{explicitly} about parallel
computation and `space--time'. Moreover, since classical mathematics is
mathematics from an omnipotent point of view, the computational model required
to capture classical mathematics, must \emph{explicitly} have unbounded
computational ability.

Since Mathematics \emph{is} computation, we, as mathematicians, write \emph{in}
a programming language. As in Computer Science, there are many programming
languages, each suited to a different paradigm, there will be numerous
mathamatical languages.

Again, in Theoretical Computer Science, given a programming language, the first
question is: \emph{How do we know what this language computes?} This question is
captured by the Denotational and Operational Semantics associated with a given
language. Plato's universe, \emph{the} Co-Algebraic universe of Mathematics,
is essentially, the fixed point of these Semantic descriptions.

When the dust settles, to define Plato's Universe, \Universe, we will need to
simultaineously define, via co--induction, the three structures of \Universe,
\Lists, \ListAutomorphisms:
%
\begin{align}
   \Universe          & \isomorphic \arrow{\ListAutomorphisms}{\Universe} & \text{Plato's Universe} \\
   \Lists             & \isomorphic \one + \Universe \times \Lists        & \text{(Gernalized) Lists} \\
   \ListAutomorphisms & \isomorphic \arrow{\Lists}{\Lists}                & \text{(Gernalized) List Automorphisms}
\end{align}
%
Unfortunately, since we \emph{are} building the foundations of Mathematics, we,
as yet, have no definitions of co--induction, $\isomorphic$, \one, $+$, $\times$,
$\arrow{\cdot}{\cdot}$, let alone, \Universe, \Lists\ or even \ListAutomorphisms.

Again, when the dust settles, Lists, \Lists, is \emph{an} \emph{implementation}
of the classical concept of the Ordinals, \Ordinal, and the isomorphism objects
of the Lists modulo List Automorphisms, \ListAutomorphisms, is \emph{an}
\emph{implementation} of the classical concepts of the Cardinals, \Cardinal. In
deed we will, eventually, make these explicit definitions, however, at the
moment, we are getting ahead of our current (minimal) abilities.

\subsection{Syntax}

There are two distinct types of references that we will use:
%
\begin{itemize}
\item Specific known references, \define{names}{}, and
\item Specific unknown references, \define{variables}{}.
\end{itemize}
%
DiSimplicial structures, which are instances of Plato's universe, are highly
structured objects with complex internal referential links, these links are
denoted using references to specific known objects using names. The `tools' we
use to build diSimplicial structures will also need to refer to other specific
diSimplicial structures which are possibly `currently' uknown. Such specific but
unknown references will use variables.


\subsection{Denotational Semantics}

\subsection{Operational Semantics}



\begin{definition}
content...
\end{definition}

\subsection{motivation}

A sketch of nearly random ideas relating to the base of the foundations of
mathematics via computational complexity.

start by looking at finite computation as above BUT we need to generalize lists
to encompase the supremum operator. See chapter 2 and 3 of Algebraic Set Theory.
compare the trees used to prove the existence of ZF-Algebras (chapter 3) with
those used to help motivate Aczel's Non-Well-Founded sets. compare these with
Vopenka's (rigid graph/tree) principle in Adamek and Rosicky's book Locally
Presentable and Accessible Categories.

Vopenka's principle is described in Mathoverflow articles:
\verb|http://mathoverflow.net/questions/29302/reasons-to-believe-vopenkas-principle-huge-cardinals-are-consistent/29473#29473|
and
\verb|http://mathoverflow.net/questions/45602/can-vopenkas-principle-be-violated-definably/46538#46538|
and or Adamek and Rosicky's book as really delineating the ``correct'' boundary
between small and large sets. As such our theory \emph{should} ensure the
Vopenka's principle is honoured.

basically using AST's existence work, we can build the ordinals just in time (?)
to extend finite computers into transfinite computers. (See Moschovakis's paper
on definition of algorithms -- he does not go far enough and use co-algebraic
methods... but his computer tools do assume ordinal computation as we need to
do).

need to understand the relationship between implicit and explicit computation
--- see the similar dichotome in the lambda-calculus.

look for p =!= np in the categorical theorems which force the use of the
small/large boundary.

look at: lucas2001transfiniteRewrite,
dershowitzKaplanPlaisted1989transfiinteRewrite,
jacobs2011coalgebraicComputation

\bibliographystyle{amsalpha}
\bibliography{diSimplexTheory}

\end{document}

