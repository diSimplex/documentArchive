% LaTeX source for the diSimplexTheory document
%

\documentclass[a4paper,openany]{amsbook}
\usepackage{perceptisys}
\usepackage{perceptisysChapter}

\begin{document}
\frontmatter
\sloppy

\title[DiSimplicial Theory]{Reality of Mathematics: Directed Simplicial Theory}
\collectionTitle{A Mathematical Theory of Reality}
\input{frontMatter}
\subjclass[2010]{Primary unknown; Secondary unknown} %
\keywords{Keyword one, keyword two etc.}%

\begin{abstract}
To be able to provide a mathematical theory of Reality, one must first 
address the question: \textit{How ``Real'' is Mathematics?}
\end{abstract} 
\maketitle 
\tableofcontents 
\mainmatter

\section{Introduction}

The aim of this cycle of papers is to provide a rigorous mathematical theory of
Reality. This first paper provides \emph{an} answer to the question: \emph{How
``Real'' is Mathematics?}, that is, \emph{what is the validity of the
mathematics used to provide a mathematical theory of Reality?}

Any answer to this question, which forms part of this cycle of papers, must of
necessity be circular: what is the Reality of the mathematics which provides a
mathematical description of Reality? How Real is mathematics? If it is not
Real/valid, then is Reality itself Real/valid?

To break this circularity, we take the pragmatic approach, bastardising
Descartes, and state: \emph{I make marks in reality, so I exist}\footnote{See
the work of the philosopher Andy Clark, \TODO{add citations}}.

We base our work on the (essential) computability of (most of) mathematics.

There are numerous schools of the foundations of mathematics.  Should we only
allow mathematics which can be computed in finite time? Should impredicative
statements be allowed? Can the full axiom of choice be allowed?  Is the
uncontrolled use of proof by contradiction acceptable? Can you do ``useful''
mathematics with out being impredicative and using the axiom of choice and
proofs by contradiction?

Restricting mathematics to \emph{only} finitely computable statements is too
restrictive. Indisciplined use of the full axiom of choice, allows mathematics
which is potentially ``unrelated'' to ``Reality''.

We take the \emph{proof relevant} approach of Martin-L\"of's dependent type
theory.

As will become obvious later in this cycle of papers, ``Reality'' is \emph{a}
model of \emph{a} higher topos.  Unfortunately, for finite beings \emph{inside}
Reality, we can only ever specify a possible collection of possible Realities. A
collection of Realities which are consistent with our current, finitely limited,
understanding, or measurement of ``Reality''.

The ``fact'' that ``Reality'' is a model of a higher topos suggests that our
chosen foundations should be ``compatible'' with the categorical approach to
mathematics. We will prove, later in this cycle of papers, that the
\emph{internal} (local) logic of a higher topos is Directed Simplicial Type
Theory (DiSiTT).

We take our lead from Homotopic type theory by being a ``proof relevant''
dependent type theoretic approach to logic. This provides a very disciplined
approach to the use of Constructivism, Intuitionism, as well as the use of
(im)predicative statements and the various versions of the axiom of choice.

Basically, we are interested in the fragment of mathematics which forms the
completion of finitely specifiable statements.

We make use of the full axiom of choice and/or uncontrolled proofs of
contradiction only to explore the pathologies outside of our finitist fragment
of mathematics.

Any discussion of the ``Reality'' or ``Validity'' of Mathematics is essentially
a discussion of the \textit{Foundations of Mathematics}.

The classical foundations of Mathematics, as developed over the past 200 years,
typically consists of some order of \emph{logic} (first or second-order)
together with the axioms of [Zermelo-Fraenkel set theory with the axiom of
choice](\verb|http://en.wikipedia.org/wiki/Zermelo-Fraenkel_set_theory|) (ZFC).

Alternate foundations \emph{can} be provided by using one or other Type Theory.
The recent [Homotopy Type Theory](http://homotopytypetheory.org/) (HoTT),
provided the final ideas required to ``crystallise'' our particular work in the
foundations of mathematics.

Current foundations of Mathematics attempt to formalise the ``Logic'' of
``Propositional/Predicate Calculus''.  Since at least 2007, and at various
levels of consciousness, I have felt that this was inadequate for the
development of a \textit{mathematics of Reality}.  However until now, I could
not clearly express how to found mathematics with out propositional/predicate
calculus.

Far more fundamental and far more important than propositions is the concept of
Causality.  I will argue, in later work, that the human brain, is a causality
inference engine. I will equally argue that the unification of Relativity and
Quantum Mechanics requires a careful understanding of the \emph{finite}
structure of causality.

As should become evident, developing the mathematics of the \emph{finite}
structure of causality, is most easily done when based upon Directed Simplicial
Theory rather than classical propositional/predicate calculus. Following the
lead of HoTT, propositional/predicate calculus, as well as set theory will be
specific ``universes'' of Directed Simplicial Type Theory.

The important point here, is that, while the mathematics of causality is
\emph{implicitly} ``part'' of classical logic, it is conceptually useful to make
the structure of causality \emph{explicit} in our foundations of mathematics. 
Inferring and manipulating causality is after all the main aim of all of the
sciences and engineering.

This book is devoted to giving the reader a more ``mathematical'' overview of
Directed Simplicial Theory.  It is meant to be read in companion with the very
much more detailed \textit{Directed Simplicial Theory Implemented in Haskell}.

\bibliographystyle{amsalpha}
\bibliography{diSimplexTheory}

\end{document}

